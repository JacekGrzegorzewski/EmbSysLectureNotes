
\qs{How are kinematics of nonholonomic robots defined?\\
How are general, affine and driftless models of kinematics defined?
}
{
    For a kinematic system, the ODE 
    describing a driftless system can be written as:
    \begin{equation}
        \dot{q} = \Sigma_{i=1}^{N}A_i(q)u_i
    \end{equation}
    If the system suffers from drint, an uncontrollable state dependant $g_0(q)$ term must be added.
    This definition is "Positive". A negative definition based on constraining a free system can 
    be written as:
    \begin{equation}
        f(q,\dot{q}) = 0
    \end{equation}
    If these constraints can be written as:
    \begin{equation}
        A(q)\dot{q} = 0
    \end{equation}
    They are called Pffafian.
    Usually the positive definition is more usefull.

}


\qs{What are standard models of wheeled robots?}
{
    There are several models of wheeled robots. The simpelest example is the unicycle, which
    is usually represented either using the differential drive model, or angular/linear velocity model.\\
    A more complicated example is the kinematic car, which has front and back wheels that act somewhat indepndantly.
    More complicated models can be generated from theese, such as chains of trailer trailing behind a car.
}

\qs{What are definitions of fundamental concepts in probability?}
{
%    
%    Axioms
%I 0 ≤ p(X) ≤ 1
%I p(T rue) = 1 p(F alse) = 0
%I p(X ∪ Y ) = p(X) + p(Y ) − p(X ∩ Y )
    The axioms of probility state, that the probability of an event
    is a real number between 0 and 1, with 1 describing events that are
    certain to occur. Obciously $P(x)+ P(!x) = 1$.\\
    A random variable $X$ is a function which maps events from a given set
    to real numbers from 0 to 1. The probability that a given event
    has a certain value can be written as $P(X = x_k)$.
    The expected value of a random variable is defined for discrete
    probabilities as:
    \begin{equation}
        E(x) = \Sigma x P(x)
    \end{equation}
    While for continuous:
    \begin{equation}
        E(x) = G\nt xP(x) dx
    \end{equation}  
    \\
    Entropy is defined as:
    \begin{equation}
        H_p(x) = E(-\log_2{p(x)})
    \end{equation}
}

\qs{How are normal distributions defined and parameterized?}
{
    \begin{equation}
        \mathcal{N}(\mu,\Sigma) \sim p(X) = \frac{1}{\sqrt{(2\pi)^{n}\text{det}\Sigma}}e^{-\frac{1}{2}(x-\mu)^{T}\Sigma^{-1}(x-\mu)}
    \end{equation}

}
