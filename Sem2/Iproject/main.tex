\documentclass{report}


\input{tex/preamble}
\input{tex/macros}
\input{tex/letterfonts}


\title{\huge{Biquaterionic polynomial decomposition,\\
and its application to kinematics and manipulator synthesis}\\
Jacek Grzegorzewski}

\author{\Huge{Intermediate Project}\\ Ph.D. Witold Paluszyński\\ KCiR WUST}

\includeonly{%
             tex/preabmle,
             tex/macros,
             tex/letterfonts,
             chapters/chapter_1,
             chapters/chapter_2,
             chapters/chapter_3,
             chapters/milestone_1,
             chapters/final
             }



\begin{document}
\begin{minipage}[h!]{0.8\textwidth}
    \maketitle
\begin{abstract}
Biquaterionic polynomials can be a powerful language in describing motion in SE(3). These mathematical constructs have the interesting property, that
their decomposition defines a topology of a mechanism - planar, spherical, or spatial - and as such could serve as a convenient method of manipulator design and analysis. In fact, they could serve as an extension of local Lie Algebraic methods, which describe a given manipulator's motion by a so called "virtual chain". Based on available literature, a bivariate motion polynomial of degree 3 will be designed to describe a desired motion , and decomposed into linear factors. Possible ways of turning this decomposition into a kinematically viable mechanism are then described. It was found that the multivariate case presents multiple problems not seen in the univariate one, and the particular form of the factors into which a motion polynomial is decomposed is quite restrictive. In the future these assumptions will have to be loosened to allow for the use of this decomposition method in practical kinematic design problems.
\end{abstract}
\vspace*{\fill}
\begin{flushright}
\Huge{All rights reserved \textcopyright}
\end{flushright}
\end{minipage}


%\newpage% or
\cleardoublepage
% \pdfbookmark[<level>]{<title>}{<dest>}
\pdfbookmark[section]{\contentsname}{toc}
%\tableofcontents
\pagebreak

%\chapter{Motivation}
\section{Parallel mechanisms}

\dfn{Parallel Kinematic Chain}
{
    A purely parallel kinematic chain is a mechanism, in which multiple serial chains are connected to the end effector.
    For the sake of this presentation however, hybrid manipulators will also be considered parallel.\\
    A hybrid manipulator is any kinematic chain containing closed transformation loops.\\
    The simplest example of a parallel mechanism is the planar 4-bar:
    %img
    \\
    However more complex structures are also commonly employed in robotics:
    %
    %delta i gough zdjecia
}
\subsection{Characteristics}
\begin{itemize}
        \item Mass concentrated near the base
        \item Higher stability
        \item Complex workspaces
        \item Difficult mathematical description
\end{itemize}

\subsection{Analysis}
\begin{itemize}
        \item Mostly vector based
        \item Often exploit symmetry
        \item Analytic solutions to problems are rare
        \item More types of singularities
\end{itemize}

\subsection{Synthesis}
\begin{itemize}
        \item Mostly based on ingenuity
        \item Systematic methods are still trapped in the serial-chain framework
        \item Often exploit Lie sub algebras of the displacement group
        \item Generally lacking
\end{itemize}
\subsubsection{Virtual chain method}
\dfn{Virtual chain method}
{
    The Virtual chain method was developed by dr. Kong and dr. Gosselin for the description of the desired motion of the end effector of the synthesised manipulator.\\
    It worked by describing a serial manipulator following a desired motion, and then constructing parallel legs of an appropriate type to restrict the motion of this virtual manipulator. This was done using Lie Algebraic methods, by creating a catalogue of serial chains, along with the wrenches constricting them.
    The intersection of these constraints should at the end produce the desired motion.
}


\section{Deficiencies}

The Virtual chain method suffers from the clear issue of needing an a priori description of the desired motion, without having a well described apparatus to that purpose. 
Even while describing their method, they described 2 types of motion which could not be realised by serial chains, but gave no systematic way to arrive at them. This ultimately makes the task of synthesising a parallel mechanism require a large deal of ingenuity, which greatly restricts their use to a few well described examples.




%
\chapter{What 2023-10-12}
\section{Random sequences}
\dfn{Limit}
{
    sequence $x_1,x_2,\cdots ,x_N$ is said to approach a limit g if for an arbitrary $\epsilon$ we can always find an  $n_0$ after which the difference between the limit and the sequence terms is less than this $\epsilon$, which we write as follows:
    \begin{equation}
        \begin{aligned}
            \lim_{N\rightarrow \infty} x_N = g &\leftrightarrow \forall \epsilon < 0 \; \exists n_0 \; \forall N > n_0 \;\; \norm{x_n-g}   < \epsilon &
        \end{aligned}
    \end{equation}
    This definition however cannot be used with random variables as the norm condition may or may not hold. So we need a new definition for random sequences.

}
\dfn{Weak Convergence}
{
    \begin{equation}
        \begin{aligned}
            x_N \rightarrow g &\leftrightarrow P(\norm{x_N-g} > \epsilon) &\rightarrow 0&\mbox{}\\[1.25ex]
                              &\leftrightarrow P(\norm{x_N - g} \le \epsilon) &\rightarrow 1&\mbox{}\\[1.25ex]
                              &\leftrightarrow \lim_{N \rightarrow \infty}(\norm{x_N - g}\le \epsilon) &= 1&\mbox{}\\[1.25ex]
            
        \end{aligned}
    \end{equation}
    Weak convergence does not guarantee that certain events won't happen after some point in the sequence, it only speaks about probabilities at infinity.
}
\dfn{Strong Convergence}
{
    \begin{equation}
        P(\norm{x_N-g} > \epsilon)& \rightarrow 0&\mbox{  Faster than $\sim \frac{1}{N}$, which implies $\Sigma^{\infty}_{n_0}P(\norm{x_N -g} > \epsilon) < \infty$}\\[1.25ex]
    \end{equation}
    Curiously then, we get $P(\lim_{N \rightarrow \infty}x_N = g) = 1$
    some wierd notation with super and subtext over the limit arrow
    
}
\ex{Difference between 2 definitions}
{
    Let \[
        x_N = \begin{cases}
            1, \; \text{with probability } 1-\frac{1}{N}\\
            N, \; \text{with probability} \frac{1}{N}
        \end{cases}
    .\]
    As N goes to infinity, $x_N$ goes weakly to 1, because the other case will still happen with non zero probability no mater the index. The strong limit on the other hand does nto exist, the convergence is too slow.
}

\dfn{Mean Square consistancy}
{
    to be continued...

}


\nt
{
    Both the means square consistancy and strong consistancy imply weak consistancy. But niether strong nor mean square consistancy imply anything about each other.
}

\begin{myproof}
    Mean square consistancy($L_2$) implies weak consistancy, which can be proven as follows:\\
        \begin{equation}
            \begin{aligned}
                E(x_N-g)^{2}&\equiv \int_{\omega in \Omega}(x_N-g)^{2}d\omega&\mbox{Because we're integrating a square, integrals over all other subsets are smaller}\\[1.25ex]
                \int_{\omega \in \mathbb{O} \subset \Omega}(x_N-g)^{2}&\text{  s.t. }\mathbb{O} = \{x_N \in \mathbb{O} \: \norm{x_N-g} > \epsilon \}&\mbox{}\\[1.25ex]
                \int_{\omega \in \mathbb{O} \subset \Omega}(x_N-g)^{2}&\ge \int_{\mathbb{O}}\epsilon^{2}d\omega&\mbox{}\\[1.25ex]
                \epsilon^{2}\int_{\mathbb{O}}1d\omega&=\epsilon^{2} P(\norm{x_N -g} > \epsilon)&\mbox{The integral is the probability measure of the set}\\[1.25ex]
            \end{aligned}
        \end{equation}
        We 've determined then that $P(\norm{x_N -g} > \epsilon) \le \frac{1}{\epsilon^{2}}E(x_N-g)^{2}$, which proves the original statement.
\end{myproof}


\thm{Strong law of large numbers(Kolmogorov $<3$)}
{
    Each theorem has some assumptions and a thesis, which should follow from them. The assumptions of this theorem are as follows:
    \begin{itemize}
            \item We have a sequence of random variables which are independant and identically distributed.
            \item The expected value of $X_i$ is $E(X_i) = m$, and the variance  $\text{Var}({X_i}) = v < \infty$ has to exist and be less than infinity.
    \end{itemize}
    The thesis then is the following equation:
    \begin{equation}
        \label{SLN_Kol}
        \frac{1}{N}\Sigma_{i=1}^{N}X_i \xrightarrow{p.1} m,\; \text{As N}\rightarrow \infty
    \end{equation}
}


\thm{Central Limit Theorem (Lindenberg-Levy)}
{
    Same assumptions here as for the strong law of large numbers(Can i reference theorems???? Check in the future)
    The implication is:
    \begin{equation}
        \frac{1}{N}\Sigma^{N}_{i=0}X_i \underset{N\rightarrow \infty}{\rightarrow} \mathcal{N}(m,\frac{v}{N})
    \end{equation}
}
\ex{Simple example}
{
    Let's take $X_i = \text{rand}() - 0.5$, the expected value is  $m = E(X_i) = \int_{-0.5}^{0.5}xdx = 0$, and the variace $\text{var}(X_i) = E(X_i^{2}) = \frac{1}{12}$.\\
    If we take $X_1, X_2, \cdots , X_{12} \approx U[-\frac{1}{2},\frac{1}{2}]$, then we havve:
    \begin{equation}
        \Sigma_1^{12}X_i \approx \mathcal{N}(0,1)
    \end{equation}

}

%\chapter{ROS2 - part 2}
Content:
\begin{itemize}
    \item Communication mechanisms: services and actions
    \item Nodes execution
    \item Configuration, launchers
    \item Transformations: tf2
    \item Debugging tools: rqt\_plot, view\_frames
\end{itemize}
\section{Communication}
\subsection{Services}
\begin{itemize}
    \item Request-response model (one-to-one)
    \item Services defines as .srv files
    \item Underlaying layer: DDS
    \item Standard services: std\_srvs package

\end{itemize}



\subsection{Actions}

\begin{itemize}
        \item Uses a client-server model
        \item Functionality similar to services
        \item Preeemptable(can be cancelled during execution)
        \item Provides feedback during the execution
        \item An action client sends a goal to an action server that acknowledges the goal and returns a feedback result
        
\end{itemize}



\subsection{Comparison}
Topics:
\begin{itemize}
        \item Purpose: continuous data streams, e.g. sensor data
        \item Many to many connection
        \item Data might be published and subscribed independently (at any time)
\end{itemize}
Services:
\begin{itemize}
        \item Purpose: remote procedure calls that can be executed quickly e.g. getting the current battery state
        \item One to one connection
\end{itemize}
Actions:
\begin{itemize}
        \item Purpose: remote procedure calls that runs for a longer time but provides feedback during the execution e.g. robot movement
        \item One to one connection, can be preempted
\end{itemize}


\section{Nodes parametrization}

\begin{itemize}
        \item Supported types:
            \begin{itemize}
                    \item bool, int64, float64, string
                    \item byte[], bool[], int64[], float64[], string[]
                    
            \end{itemize}
        \item Defined in YAML files
        \item Parameters are node specific (different than in ROS1 where it was a parameter server)

\end{itemize}

\nt
{
    Useful commands(CLI)
    \begin{itemize}
            \item ros2 param list
            \item ros2 param get <node\_name> <param\_name>
            \item ros2 param set <node\_name> <param\_name> <value>
    \end{itemize}
}


\section{Launchers}
\begin{itemize}
    \item The \textbf{launch system} in ROS2 is responsible for helping the user describe the configuration of their system and then execute it as described. The configuration of the system includes what programs to run, where to run then, what arguments to pass them.
    \item It's a script that runs the whole robotic system, so it allows to start multiple nodes with one command instead of starting each node manually
    \item Can be written in Python, XML, YAML
    \item CLI: ros2 launch package\_name launcher\_name
\end{itemize}


\section{Executors}


Execution management in ROS 2 is explicated by the concept of Executors. An Executor uses one of more threads of the underlying operating system to invoke the callbacks of subscriptions, timers, service servers, action servers, etc. on incoming messages and events.
\subsection{what?}
When the spin() function is called, the current thread starts querying the rcl and middleware layers for incoming messages and other events, and calls the corresponding callback functions until the node shuts down.
An incoming message is not stored in a queue on the Client Library Layer but kept in the middleware until it is taken for processing by a callback function. 
A wait set is used to inform the Executor about available messages on the middleware layer, with one binary flag per queue.
\subsection{Executor types}

\subsubsection{The Multi-Thread Executor}
The Multi-Thread Executor creates a configurable number of threads to allow for processing multiple messages or events in parallel.

\subsubsection{Static Single-Threaded Executor}

The Static Single-Threaded Executor optimizes the rntime costs for scanning the structure of  a node in terms of subscriptions, timers, service servers, action servers, etc.
It performs this scan only once when the node is added, while other executors regularly scan for such changes.




\section{TF2}

\subsection{Rottaion representation - RPY angles}

\begin{itemize}
        \item RPY $\rightarrow$ roll, pitch, yaw
        \item ZYX euler angles
        \item Three values representation
        \item drawback: singular configurations (system lose one DOF) if two axes are aligned
        
\end{itemize}


dupa



\subsection{Rotation representation - Unit quaternions}

A quaternion is a 4-tuple representation of orientation, which is more concise than a rotation matrix, Quaternions are very efficient for analysing situations where rotations in three dimensions are involved.

 \begin{equation}
    q = w + xi + yj + zk
\end{equation}

\subsection{Rotation representation - Rotation matrix}

A rotation matrix is a transformation matrix that is used to perofrm a rotation in Euclidean space
\begin{item}
    $R \in \mathbb{R}^{3\times 3}$ 
\item $R R^{T} = R^{T}R = \mathbb{I}_3$ 
\item $det(R) = 1$
\end{item}

\subsection{tf2 in ROS2}

tf2(team fortress 2) is a library that provides support or managing coordinate systems in a robotic system.
\begin{itemize}
        \item Provides support for handling relationships between different coordinate frames.
        \item Essential for most robotic tasks, such as sensor fusion, localization
        \item Enables easy conversion of data between various frames of reference
\end{itemize}

\begin{itemize}
        \item Transofrmation = translation + rotation(quaternion) WHY NOT JUST A BIQUATERNIONS, NOW WE HAVE TO CONSTRUCT UNNECESSARY PAIRS OF QUATERNIONS
        \item Coordinate system $\rightarrow $ frame
        \item Components: Dynamic Broadcaster, Static Broadcaster, Listener
\end{itemize}

\subsection{Dynamic TF Broadcaster in ROS2}

Dynamic TF broadcasting allows real-time updates to coordinate transforms, providing flexibility in robotic systems.
\begin{itemize}
        \item Dynamically updates transform data during runtime
        \item Useful for scenarios, where relationships between frames change dynamically (e.g. moving robot)

\end{itemize}

\subsection{Static TF Broadcasting in ROS2}

Static TF broadcasting in ROS 2 is used to establish fixed coordinate transforms between frames that remain constant throughout the robot's operation.
\begin{itemize}
        \item Defines transforms that do not change during runtime.
        \item Ideal for representing fixed relationships between frames in a robotic system, e.g. transform between sensros
        \item Improves efficiency by removing the need for continuous updates

        
\end{itemize}

\subsection{TF Listening in ROS 2}
TF listening in ROS 2 allows a node to receive and use real-time coordinate transforms between frames, providing up-to-date spatial information in a robotic system.

\begin{itemize}
        \item Listens to TF messages to retrieve transforms dynamically
        \item Essential for tasks requiring knowledge of spatial relationship in real-time
        \item Enables adaptability in response to changes in the robot's environment.        
\end{itemize}


\section{ROS tools - rqt\_graph}
\begin{itemize}
        \item rqt\_graph can visualize the ROS graph of the application
        \item The ROS graph contains all the running nodes etc.
\end{itemize}

A
\section{ROS tools - view\_frames}
\begin{itemize}
        \item The tool can be used to visualize the transformation tree of the system
\end{itemize}



%\chapter{November Milestone}
The goal of this project is to develop an algorithm for factoring a multivariate biquaterionic polynomial into factors corresponding to elementary
joints. For the november milestone the following subgoals have been acomplished:
\begin{enumerate}
        \item Describing serial chains using biquaterionic polynomials
        \item Development of a division algorithm for biquaterionic polynomials
        \item Initial investigation of the kinds of polynomials corresponding to mechanical joints in the multivariate case.
        
\end{enumerate}
As the first 2 subgoals were discussed in previous chapters, and as such will be covered briefly.
\section{Description of serial chains}
Although the Denavit-Heartenberg convention can be readily translated into
the language of biquaternions, it was found that it's more natural to use joint axes instead to describe the geometry of a given manipulator. 
While not a minimal description - the Denavit-Heartenberg convention is considered minimal as it reqieres only 4 parameters to describe a frame transforation - it gives insight into how a result of decomposition might look.
\\
As an ilutration, a planar 2R manipulator was described using linear
biquaterionic polynomials. The following motion polynomials
were obtained:
        \begin{equation}
            \begin{cases}
                R_1(t) = 1 + tk\\
                R_2(u) = (1-\varepsilon\frac{L_1}{2}i)(1+uk)(1+\varepsilon\frac{L_1}{2}) = 1+uk + u\varepsilon L_1j
            \end{cases}
        \end{equation}
        The product of which gives:
        \begin{equation}
           R(t,u)= R_1(t)R_2(u) = 1-tu +(t+u)k + \varepsilon L_1(sj-ti)
        \end{equation}
This bivariate motion polynomial can act on the end effector located at $1+\frac{\varepsilon}{2}(L_1+L_2)i$ for the initial configuration, producing:

   \begin{equation}
   \begin{cases}
        x' = 1 + \varepsilon\frac{Px\bar{P} + 2P\bar{Q}}{P\bar{P}}\\
        Px\bar{P} = (L_1+L_2)((1-ts)^2 - (s+t)^2)i+2(1-ts)(s+t)j\\
        P\bar{Q} = L_1(s^2-s t^2+s t+t)i + L_1(s^2 t+s t-s+t^2)j\\
        P\bar{P} = (1+t^2)(1+u^2)
   \end{cases}
   \end{equation}
After using the half-tangent identities to translate the result into the language of trigonometry the following forward kinematics were obtained:
   \begin{equation}
   \begin{aligned}
          x' = 1 &+ \varepsilon((L_1\cos{\theta_t} + L_2\cos{(\theta_t+\theta_u)})i\\
        &+(L_1\sin{\theta_t} + L_2\sin{(\theta_t+\theta_u)})j)
   \end{aligned}
   \end{equation}
Which are in agreement with ones found through other means.


\section{Division algorithm}
A division algorithm for multivariate polynomials requieres defning an ordering to its various monomial terms\cite{cox}. For a different ordering, different results will be obtained. Algorithms for a commutative field can be extended into the non-commutative case, by noting that there are now 2 division algorithms, a left and a right one. Both of these were described before, and were found to work as intended under the assumption that the indeterminates commute with the coefficients.
\\
\noindent\begin{minipage}{.5\linewidth}


\begin{algorithm}[H]
\KwIn{f1,f}
\KwOut{q,r}
\SetAlgoLined
\SetNoFillComment
\vspace{3mm}
$q \leftarrow 0; r \leftarrow 0$\\
$p \leftarrow f$\;
\While{p \neq 0}
{
    divisionoccured \leftarrow false\\
    \uIf{$LT(f_1)$ divides $LT(p)$}
    {
    $q \leftarrow q + LT(p) LT^{-1}(f_1)$\\
    $p \leftarrow p - LT(p) LT^{-1}(f_1)f_1$

    divisionoccured \leftarrow true\\
    }
    \uIf{ divisionoccured == false}
    {
        $r \leftarrow r + LT(p)$\\
         $p \leftarrow p - LT(p)$
   }
}
\Return q,r\;
\caption{Left division}
\end{algorithm}

\end{minipage}
\begin{minipage}{.5\linewidth}


\begin{algorithm}[H]
\KwIn{f1,f}
\KwOut{q,r}
\SetAlgoLined
\SetNoFillComment
\vspace{3mm}
$q \leftarrow 0; r \leftarrow 0$\\
$p \leftarrow f$\;
\While{p \neq 0}
{
    divisionoccured \leftarrow false\\
    \uIf{$LT(f_1)$ divides $LT(p)$}
    {
    $q \leftarrow q + LT^{-1}(f_1)  LT(p) $\\
    $p \leftarrow p -  f_1LT^{-1}(f_1) LT(p)$

    divisionoccured \leftarrow true\\
    }
    \uIf{ divisionoccured == false}
    {
        $r \leftarrow r + LT(p)$\\
         $p \leftarrow p - LT(p)$
   }
}
\Return q,r\;
\caption{Right division}
\end{algorithm}


\end{minipage}

As an example, the result of left division of the polynomial:
\begin{equation}
    f = 1 +ti + uj + tuk = (1 + ti)(1+uj)
\end{equation}
by $1+uh$ is:
\begin{equation}
    f= (1+uh)(th^{-1}k + h^{-1}j) + t(i - h^{-1}k) + 1 - h^{-1}j
\end{equation}
%%
Which has no remainder only if $h = j$

Something to keep in mind is that this remainder will be different depending on the chosen monomial ordering. This is concerning, as this gives the zeroing of the remainder as a sufficient, but not necessary condition for f1 to be a factor of f. For this algorithm to be useful then, one would need to find an appropriate order for agiven factor beforehand to zero its remainder. 


\section{Polynomial-joint correspondance}
This goal was not described previously, as it is still not fully entierly achieved. Before one can speak of a division algorithm, one should first know what he will divide into. Firstly, for a polynomial to correspond to a mechanical joint, it must not change its geometry depending on the values of the independant variables, as an example:
\begin{equation}
    H(t,u) = 1+\varepsilon( t i + u j)
\end{equation}
The above is a linear polynomial, which can not correspond to an elementary joint. Its geometry changes with the change of variables, which can't be the case for an elementary joint. This fact limits the polynomials which
can represent elementary joints to ones which can be described as:
\begin{equation}
    H(t_1,\dots,t_n) = 1 +R(t_1,t_2,\dots ,t_n)h
\end{equation}
Where $h \in \mathbb{DH}$ and $R \in \mathbb{R}[t_1,\dots ,t_n]$. Here the coefficient of h is a real polynomial, which represents for example how far a prismatic joint is extended, or a rotational joint rotated. But as there is only one biquaternion in the polynomial, the geometry stays the same.\\

Secondly, for a given decomposition, if we fix all variables but one, we should still have a viable closed loop kinematic chain capable of 1DOF motion. Consider:
\begin{equation}
    H(t_1,\dots,t_n) = (1+R_1(t_1,\dots,t_n)h_1)(1+R_2(t_1,\dots,t_n)h_2)
\end{equation}
If we fix all the variables except for $t_1$, we will get real polynomials in  $t_1$ as coefficients of biquaternions:

\begin{equation}
    H_{t_1}(t_1) = (1+(\Sigma_{i=0}^{m}a_it^{i}_1)h_1)(1+ (\Sigma_{i=0}^{m}b_it^{i}_1)h_2)
\end{equation}
If these polynomials contain quadratic terms, then we won't obtain a properly factored kinematic chain in 1 variable and would have to perform further divisions, which contradicts the fact that a factorised chain even existed. As such, after such an evaluation the coefficients must be linear:
\begin{equation}
    H_{t_1}(t_1) = (1+(a_0_a_1t)h_1)(1+ (b_0+b_1t)h_2)
\end{equation}

Furthermore, if we evaluate a given polynomial at 0 with all variables, we should get the identity polynomial, which implies that these coefficients have their 0 degree coefficients equal to 0.\\
Ultimately, a biquaterionic polynomial corresponds to a mechanical joint if it is of the form:
\begin{equation}
    H(t) = (1+ R(t_1,\dots,t_2)h)
\end{equation}
Where $R(t_1,\dots,t_2) \in R[t_1,\dots,t_n] / (t_1^{2},t_2^{2},\dots,t_n^{0})$, $R(0,\dots,0) = 0 $ and $h \in \mathbb{DH}$.  For the first 3 free variables, these polynomial coefficients would look as follows:s

\begin{equation}
    \begin{aligned}
        R(t)&=at&\mbox{1 free variable}\\[1.25ex]
        R(t,u)&= at + bu + ctu&\mbox{2 free variables}\\[1.25ex]
        R(t,u,v)&= at +bu + cv + dtu +etv +fuv +gtuv &\mbox{3 free variables}\\[1.25ex]
    \end{aligned}
\end{equation}

This immediatly implies that unlike for the univariate case, where a left evaluation would correspond to a left factor and vice versa, here we will have $n!$ left and right evaluations.
This is because once we evaluate this polynomial at a non-comutative element, the order in which they are multiplied matters. 
It is unclear at the moment, if the order of division would have any bearing on the propper order of evalation. 
It shouldn't since the division algorithm assumes that the free variables commue anyway, but it should be kept in mind that there may be some coupling there.

\section{December plans}
\begin{itemize}
        \item Under the assumption that joints correspond to the above derived factors, the conditions for factorization will be investigated.
        \item Behaviour of different orders of evaluation will be analysed.
        \item An attempt will be made to describe a serial parallel manipulator based on the human leg for further decomposition.
\end{itemize}


        


\chapter{Final report}

\section{Introduction}
To better explain the results of later chapters, quaternions and biquaternions are first introduced, along with their kinematic interpretations.
%% Biquaternions and biquaterionic polynomials



\subsection{Quaternions}
The algebra of quaternions -- usually denoted as $\mathbb{H}$ -- is homomorphic to the 3 dimensional special orthogonal group(SO(3)), which is the group representing rotations in 3D space. A general quaternion $h$ can be written as follows:
\begin{equation}
    h = h_0 + h_1i +h_2j + h_3k
\end{equation}
The 3 unit versors $i$,  $j$, $k$ are not commutative, and subject to the following equations:
\begin{equation}
       i^{2} = -1,\;
       j^{2} = -1,\;
       k^{2} = -1,\;
       ijk = -1
\end{equation}
A quaternion with its scalar par $h_0$ equal to zero is called a purely vectorial quaternion, the set of which is denoted as $\mathcal{H}$.  The conjugate of a quaternion is defined as:
\begin{equation}
    \bar{h} = h_0 - h_1i - h_2j - h_3k
\end{equation}
The conjugate can be used to define the square of the norm of a quaternion:
\begin{equation}
    h\bar{h} = \bar{h}h = h_0^{2} + h_1^{2}+h_2^{2} + h_3^{2} \in \mathbb{R}
\end{equation}
If $h \in \mathcal{H}$, then $h + \bar{h} = 0$.
\clearpage
\subsubsection{SO(3)}
Given $v \in \mathcal{H}$, a quaternion $p \in \mathbb{H}$ can be used to rotate it around the origin by the following map:
\begin{equation}
    v' = \frac{pv\bar{p}}{p\bar{p}}
\end{equation}
In particular, if $p$ is a unit quaternion with $p\bar{p} = 1$ we have:

\begin{equation}
    v' = pv\bar{p}
\end{equation}
This is sometimes affectionately referred to as the \textbf{sandwich product}.
\\
To understand how a sandwich product of a quaternion $p$ rotates $v$, it helps to observe, that every unit quaternion can be written as:
\begin{equation}
    p = \cos{(\frac{\theta}{2})} + \sin{(\frac{\theta}{2})}u
\end{equation}
where $u\bar{u} = 1$ and  $u\in \mathcal{H}$. Then the sandwich product represents rotation of $v$ around the axis $u$ by the angle $\theta$.
To convince ourselves of this, we can first see that a rotation of $u$ around $u$ leads to the identity transformation, and investigate the behaviour of unit versors.\\
\subsubsection{Rotation of a versor lying on the rotation axis}
\begin{equation}
    \begin{aligned}
        u'&=pv\bar{p}&\mbox{}\\[1.25ex]
        u'&=(\cos{(\frac{\theta}{2})}+\sin{(\frac{\theta}{2})u})u(\cos{(\frac{\theta}{2})}-\sin{(\frac{\theta}{2})u})&\mbox{}\\[1.25ex]
        u'&=(\cos{(\frac{\theta}{2})}+\sin{(\frac{\theta}{2})u})(\cos{(\frac{\theta}{2})}u+\sin{(\frac{\theta}{2})})&\mbox{}\\[1.25ex]
        u'&=(\cos{(\frac{\theta}{2})}^{2} + \sin{(\frac{\theta}{2})}^{2})u&\mbox{}\\[1.25ex]
        u'&=u&\mbox{}\\[1.25ex]
    \end{aligned}
\end{equation}
\subsubsection{Rotation of a basis versor around another basis versor}
Let $p = \cos{(\frac{\theta}{2})} + \sin{(\frac{\theta}{2})}k$, then if we calculate the sandwich products $pi\bar{p}$ or  $pj\bar{p}$ we should expect results similar to ones well known from linear algebra:\\

\noindent\begin{minipage}{.5\linewidth}
    \begin{equation}
        \begin{bmatrix}
            \cos{(\theta)} \\
            \sin{(\theta)} \\
            0
        \end{bmatrix} = \begin{bmatrix}
        \cos{(\theta)} & -\sin{(\theta)} & 0\\
        \sin{(\theta)} & \cos{(\theta)} & 0 \\
        0 & 0 & 1
        \end{bmatrix}
        \begin{bmatrix}
            1  \\ 0 \\ 0
        \end{bmatrix}
    \end{equation}
\end{minipage}
\begin{minipage}{.5\linewidth}
    \begin{equation}
             \begin{bmatrix}
            -\sin{(\theta)} \\
            \cos{(\theta)} \\
            0
        \end{bmatrix} = \begin{bmatrix}
        \cos{(\theta)} & -\sin{(\theta)} & 0\\
        \sin{(\theta)} & \cos{(\theta)} & 0\\
        0 & 0 & 1
        \end{bmatrix}
        \begin{bmatrix}
            0  \\ 1 \\ 0
        \end{bmatrix}
    \end{equation}
\end{minipage}
And in fact:
\begin{equation}
    \begin{aligned}
         u'&=pi\bar{p}&\mbox{}\\[1.25ex]
        u'&=(\cos{(\frac{\theta}{2})}+\sin{(\frac{\theta}{2})k})i(\cos{(\frac{\theta}{2})}-\sin{(\frac{\theta}{2})k})&\mbox{}\\[1.25ex]
        u'&=(\cos{(\frac{\theta}{2})}+\sin{(\frac{\theta}{2})k})(\cos{(\frac{\theta}{2})}i+\sin{(\frac{\theta}{2})}j)&\mbox{}\\[1.25ex]
        u'&=(\cos{(\frac{\theta}{2})}^{2}-\sin{(\frac{\theta}{2})}^{2})i + 2\sin{(\frac{\theta}{2})}\cos{(\frac{\theta}{2})}j&\mbox{}\\[1.25ex]
        u'&=\cos{(\theta)}i + \sin{(\theta})j&\mbox{}\\[1.25ex]

    \end{aligned}
\end{equation}
A result in complete agreement with one obtained by the use of rotation matrices

\clearpage

\subsubsection{T(3)}

Since we already associate the 3 basis versors with the 3 standard axes of a Cartesian space, it seems natural to use quaternions to represent translations. It is in fact trivial to do, a translation is simply the addition of one purely vectorial quaternion to another:
\begin{equation}
    v' = v + u
\end{equation}
where $v,u\in \mathcal{H}$.
\subsubsection{SE(3)}
The special Euclidean group(SE(3)), which represents rigid body motions in 3D space, is composed of 2 subgroups. Those being the special orthogonal group SO(3), and the translational group in 3D T(3). Given these 2, one can construct SE(3) as the semi-direct product of them as such:
\begin{equation}
    SE(3) = SO(3)\ltimes T(3)
\end{equation}
With this construction, elements of SE(3) are given by a tuple  $(r,t)$, and a product of 2 of its elements is given as follows:
 \begin{equation}
     \label{semidirect}
     (r_2,t_2)(r_1,t_1) = (r_2r_1,r_2t_1\bar{r}_2+t_2)
\end{equation}
With $r_1,r_2\in SO(3)$, and $t_1,t_2 \in T(3)$\\
This formula should be familiar to anyone who ever multiplied 2 DHT matrices together.\\
In general then, a pair of quaternions is enough to represent the motion of rigid bodies. A problem however with this formulation is that it requires pairs of quaternions, which makes it mathematically unwieldy. This leads to the idea of enclosing both of these pairs in a single algebra.
\subsection{Biquaternions}
The way the above issue can be solved when using orthogonal matrices to represent rotation, is to embed SE(3) in SO(4). This may sound complicated, but this is in fact the most common way to represent elements of SE(3). An example would be the block matrix:
\begin{equation}
    A_i = \begin{bmatrix}
        R & T   \\
        \mathbb{0} & 1
    \end{bmatrix}
\end{equation}
Again, this should be familiar to anyone who knows anything about robot kinematics. If we multiply 2 such matrices together we will obtain:
\begin{equation}
    A_1A_2 =  \begin{bmatrix}
        R_1 & T_1   \\
        \mathbb{0} & 1
    \end{bmatrix} 
 \begin{bmatrix}
        R_2 & T_2   \\
        \mathbb{0} & 1
    \end{bmatrix} 
 \begin{bmatrix}
        R_1R_2 & R_2T_1 + T_2\\
        \mathbb{0} & 1
    \end{bmatrix}
\end{equation}
Which looks exactly like the semi-direct product presented before \ref{semidirect}. Do tho this with quaternions however, we have to utilise a different method. A way to do this, is to introduce a new type of coefficient for 
regular quaternions. Instead of using real numbers $h_i \in \mathbb{R}$ as coefficients, one can introduce dual number coefficients $h_i \in \mathbb{D}$ $h_i = p_i + \varepsilon q_i$, where  $\varepsilon^{2} = 0$. By using these dual coefficients, we create a new algebra called the dual quaternions(sometimes also known as bi quaternions) which we will denote as $\mathbb{DH}$. An element $h \in \mathbb{DH}$ can also be written as 
 \begin{equation}
    h = p + \varepsilon q
\end{equation}
Where $p,q \in \mathbb{H}$. 
A necessary condition for a biquaternion to describe an element of SE(3) is that it lies in the Study quadric\cite{Heged_s_2013}. This condition for a biquaternion can be alternatively stated that it needs to have a real non-zero norm. A norm of a biquaternion is defined in terms of its conjugate as follows:
\begin{equation}
    \begin{aligned}
        h\bar{h}=(p+\varepsilon q)\bar{(p+\varepsilon q)}\\
        h\bar{h}=(p+\varepsilon q)(\bar{p}+\varepsilon \bar{q})\\
        h\bar{h}=p\bar{p} + \varepsilon (p\bar{q} + q\bar{p}) 
    \end{aligned}
\end{equation}
For the above to be a real number, the dual part needs to be zero, e.g. $  (p\bar{q} + q\bar{p}) = 0$. The rotation can be extracted by just taking the primal part p of the biquaternion. Its action upon elements of SE(3) is defined the same as for regular quaternions. The position of a point in space - represented as $x = 1+\varepsilon v$, where v is a vectorial quaternion - is given by the map\cite{Siegele_2021}:
\begin{equation}
     x \rightarrow \frac{(p - \epsilon q)x(\bar{p}+\epsilon \bar{q})}{p\bar{p}}
\end{equation}
Which can be rewritten:
\begin{equation}
    \label{act}
    \begin{aligned}
        \frac{(p - \epsilon q)x(\bar{p}+\epsilon \bar{q})}{p\bar{p}}&=\frac{px\bar{p} + \epsilon(px\bar{q} - qx\bar{p})}{p\bar{p}}&\\[1.25ex]
        \frac{px\bar{p} + \epsilon(px\bar{q} - qx\bar{p})}{p\bar{p}}&= 1+\epsilon\frac{pv\bar{p} + p\bar{q} - q\bar{p}}{p\bar{p}}&\\[1.25ex]
        1+\epsilon\frac{pv\bar{p} + p\bar{q} - q\bar{p}}{p\bar{p}}&= 1 + \epsilon \frac{pv\bar{p} + 2p\bar{q}}{p\bar{p}}&\\[1.25ex]
    \end{aligned}
\end{equation}

Note that the map does not require the biquaternion to be of unit norm. In fact, the mapped point remains the same no matter what real number we multiply the polynomial by. By noticing this, we can treat the space of rigid body motions as a projective space, allowing for the use of polynomials to represent trajectories in it. The simplest such motions are translations along a line, and rotation around a point. Both can be described using linear biquaterionic polynomials.

\subsubsection{Elementary motions}
A linear biquaterionic polynomial can represent either a translation, or a rotation. 
A general formula for a linear univariate biquaterionic polynomial parametrising a rigid body motion is the following:
\begin{equation}
    F = 1+th,\; h \in \mathbb{DH}
\end{equation}
To represent a motion in SE(3), its norm polynomial has to be real:
\begin{equation}
    F\bar{F} = 1+(h+\bar{h})t+h\bar{h}t^{2} \in \mathbb{R}[t]
\end{equation}
Since the parameter t is always real, this translates to:
\begin{equation}
    \begin{cases}
        h\bar{h} \in \mathbb{R}\\
        h+\bar{h} \in \mathbb{R}
    \end{cases}
\end{equation}
Since $h = p + \epsilon q,\; p,q \in \mathbb{H}$, this is equivalent to saying that $h$ itself has to satisfy the Study condition -- and so represent an element of SE(3) -- and has to have purely vectorial dual part, e.g.
\begin{equation}
    h = p_0+p_1\mathit{i}+p_2\mathit{j}+p_3\mathit{k} +\epsilon(q_1\mathit{i}+q_2\mathit{j}+q_3\mathit{k})
\end{equation}
\subsubsection{Elementary motions -- translation}
Translations form a subgroup of SE(3), and can be easily represented by a purely dual biquaternion. In general such a quaternion has the form:
\begin{equation}
    h_t = 1 - \frac{\epsilon}{2}(q_1\mathit{i}+q_2\mathit{j}+q_3\mathit{k})
\end{equation}
Its action will translate a given point $x = 1 + \epsilon(v)$ along the vector $q$ according to \ref{act}:
\begin{equation}
    \begin{aligned}
        x &\rightarrow 1 + \epsilon\frac{pv\bar{p} + 2p\bar{q}}{p\bar{p}}&\mbox{Note that $p = 1$ and  $q = q_1\mathit{i}+q_2\mathit{j}+q_3\mathit{k}$}\\[1.25ex]
        x&\rightarrow 1 + \epsilon(v + q)&\mbox{The point is mapped onto its original position plus the translation vector}\\[1.25ex]
    \end{aligned}
\end{equation}

We can parametrise this map using a real parameter $t$, thus giving rise to a linear biquaterionic polynomial parametrisation of translations in SE(3):
\begin{equation}
    H_t(t) = 1 - \frac{\epsilon t}{2}q
\end{equation}
It maps a point x -- defined as before -- as follows:
\begin{equation}
    x \rightarrow  1 + \epsilon v + \epsilon qt
\end{equation}
The parameter t specifies where along the line defined by the vector q the point is translated. For $t = 0$ we get the identity transformation.
Since $h_t$ satisfies the Study condition, so does  $H_t$, and thus we've associated translations with linear trajectories in the study quadric.
\subsubsection{Elementary motions -- rotations}
It is well known, that rotations around the origin can be represented by unit quaternions. This representation can easily be extended to the projective case, and gives an intuitive interpretation of the indeterminate parametrising the rotation. A quaternion representing a rotation about an axis $q$ located at the origin is given by the following:
\begin{equation}
    \begin{aligned}
        q&= \cos{\theta} + \sin{\theta}(q_1\mathit{i} + q_2\mathit{j} + q_3\mathit{k})&\mbox{Divide by $\cos{\theta}$}\\[1.25ex]
        \frac{q}{\cos{\theta}}&\equiv q&\mbox{$\equiv$ we take to mean projectively equal}\\[1.25ex]
        q&\equiv 1 + \tan{\theta}(q_1\mathit{i} + q_2\mathit{j} + q_3\mathit{k})&\mbox{}\\[1.25ex]
    \end{aligned}
\end{equation}
By interpreting $\tan{\theta}$ as the indeterminate variable t, we can represent rotation around an axis at the origin as a linear quaterionic polynomial. \\
To represent a rotation around an axis located at another location, the translation biquaternion can be used to translate the rotation quaternion\cite{Siegele_2021}. To do this, the same map \ref{act} as the one used to represent the movement of a point can be used, but instead of a point it will act on a rotation quaternion. 
\begin{equation}
    \label{rottr}
    \begin{aligned}
        (1-\epsilon \frac{x}{2})(1+tv)(1+\epsilon \frac{x}{2}))   &= 1 + tv +t\frac{\epsilon}{2}(vx-xv)&\mbox{}\\[1.25ex]
        P(t)&= 1 + tv&\mbox{The primal part}\\[1.25ex]
        Q(t)&= \frac{t}{2}(vx-xv)&\mbox{The dual part}\\[1.25ex]
    \end{aligned}
\end{equation}
The above map describes a rotation around an axis $v$, which goes through a point $x$. To see that this is the case, we can show that it's action is not going to affect any point which lies on the axis it rotates around. The coordinates of all such unaffected points will lie on a line which can be represented as a biquaternion $z = 1 + \epsilon(x + u v)$, where  $u\in \mathbb{R}$, and both x and v are purely vectorial quaternions, with $v\bar{v} = 1$. Then the point $z$ will be mapped to another point by \ref{rottr} according to \ref{act}:
\begin{equation}
    z \rightarrow 1 +\epsilon \frac{P(t)(x+uv)P(t) + 2P(t)\bar{Q}(t)}{P(t)\bar{P}(t)}
\end{equation}
The second part can be expanded as:
\begin{equation}
    \begin{aligned}
        2P(t)\bar{Q}(t)&=2(1+tv)\frac{t}{2}(xv-vx)&\mbox{}\\[1.25ex]
        2P(t)\bar{Q}(t)&=t(xv-vx)+t^{2}(vxv - x)&\mbox{}\\[1.25ex]
    \end{aligned}
\end{equation}
While the first as:
\begin{equation}
    \begin{aligned}
        P(t)(x+uv)P(t)&=P(t)x\bar{P}(t) + u P(t)v\bar{P}(t)&\mbox{}\\[1.25ex]
        uP(t)v\bar{P}(t)&= u(1+t^{2})v&\mbox{}\\[1.25ex]
        P(t)x\bar{P}(t)&= x + t(vx-xv)-t^{2}vxv&\mbox{}\\[1.25ex]
    \end{aligned}
\end{equation}

Adding both of these together we get:
\begin{equation}
    P(t)(x+uv)\bar{P}(t) + 2P(t)\bar{Q}(t) = x + t(vx-xv) - t^{2}vxv+u(1+t^{2}) + t(xv-vx) + t^{2}(vxv-x)
\end{equation}
Which after simplification yields:
\begin{equation}
    P(t)(x+uv)\bar{P}(t) + 2P(t)\bar{Q}(t) = (x + uv)(1+t^{2})
\end{equation}
Which ultimately shows that this rotation polynomial maps z to itself:
\begin{equation}
    z \rightarrow 1+\epsilon(x+uv)
\end{equation}

\subsection{Bivariate polynomials}
Extending this to the bivariate case, there are a couple of possibilities as to what will serve as the analogue of the linear elementary motions. For a serial chain with every joint being independent, there is no such problem, and we can simply take the new linear terms to be in either of the free variables. This assumption simplifies calculations by a large margin, but restricts the class of motion polynomials under consideration to those who's norm can be written as a product of 2 polynomials in 2 different variables. This latter approach was ultimately the one chosen, however a discussion of the alternative choice will follow later on in the report.\\
We are now in a position to state the design procedure.
\begin{itemize}
    \item Describe a desired 2 DOF motion using a motion polynomial. This serves as an analogue to the virtual chain of Dr Gosselin's method\cite{gosselin}.
    \item Factorize this motion polynomial, to produce the following representation:
        \begin{equation}
    H(u,t) = (1-h_0 v_0)(1-h_1 v_1)\dots(1-h_nv_n)
        \end{equation}
Where:
\begin{itemize}
        \item $h_i$ - is a motion biquaternion describing a rotation
        \item  $v_i \in \left\{ u, t \right\}$ - is a free variable taking on real values       
\end{itemize}
There may be multiple such factorizations, each of which represents a serial kinematic chain realising the desired motion.
\item Somehow constrain these chains to restrict the degrees of freedom of each mechanism to the number of free variables in the motion polynomial. Thus creating the desired mechanism.
\end{itemize}


\section{Methods}
Despite the assumed simplifications, the problem is still very complex.
To simplify computations, another simplification was made based on the observation, that every motion in $SE(3)$ has a spherical projection in  $SO(3)$. This is because rotation is not changed by a translation, so for example a 6 DOF manipulator may be performing very complex motions, the orientation of the end effector is just the result of multiplying individual joint rotation matrices. 
This means, that every spatial mechanism can be reduced to a spherical one containing only rotation, and that a spherical mechanism can be turned into a spatial one by translating its rotation axes.\\
In case of biquaterionic polynomials, this means that just the product of
the quaterionic parts of every polynomial defines a spherical mechanism. What's more its decomposition will be the same as the spatial one if one removes the dual part from every term in the product. To illustrate this process, a motion polynomial describing the spatial motion of a Bennet mechanism is considered\cite{li2015factorization} :
\begin{equation}
    C(t) = 1 +  (1-j)t + (1-i-j-k)t^{2} - \varepsilon((i-j-k)t-(1+k)t^{2})
\end{equation}
This mechanism admits 2 factorizations:
\begin{equation}
    \begin{cases}
        C(t) = (1-t(j+k+\varepsilon(i+j-k)))(1+t(1+k+2\varepsilon j ))\\
        C(t) = (1+t(1+i+\varepsilon k))(1-t(i+j+\varepsilon(i-j))
    \end{cases}
\end{equation}
The spherical projection is obtained by taking the real part of each factor, and the result is the real part of $C(t)$, which is a viable quaterionic motion polynomial, and as such represents a spherical mechanism.

\begin{equation}
    PrimalPart(C(t)) = 1 +  (1-j)t + (1-i-j-k)t^{2}
\end{equation}
This mechanism admits 2 factorizations:
\begin{equation}
    \begin{cases}
        PrimalPart(C(t)) = (1-t(j+k))(1+t(1+k))\\
        PrimalPart(C(t)) = (1+t(1+i+k))(1-t(i+j))
    \end{cases}
\end{equation}
This ultimately means, that as long as this method works for quaterionic polynomials, it will work for biquaterionic ones as well.

\subsection{Desired motion}
The motion polynomial considered was designed to rotate the z axis to a position described by spherical coordinates:
\begin{equation}
    qk\bar{q} \rightarrow i \sin{\theta}\cos{\psi}  +  j \sin{\theta}\sin{\psi}  + k\cos{\theta}
\end{equation}

\begin{figure}[h!]
    \centering
    \includegraphics[scale=0.3]{img/831px-3D_Spherical_2.png}
    \caption{Visualisation of the \href{  https://en.wikipedia.org/wiki/Spherical_coordinate_system }{spherical coordinate frame} in relation to the cartesian one.}
    \label{fig:enter-label}
\end{figure}

There is no one single way to find a polynomial transforming one vector into the other. A way to do it involves finding the vector cross product of the 2 vectors - say $v$ and  $u$ - to represent the rotation axis, and their dot product to determine the angle of rotation:
\begin{equation}
    q = v\cdot u + \sqrt{u\bar{u}v\bar{v}} + v \times u
\end{equation}
Applying this method to the 2 vectors in the map given at the beginning we obtain:
\begin{equation}
    q = 1 + \cos{\theta} + \sin{\theta}(\cos{\psi}j-\sin{\psi}i)
\end{equation}
This polynomial does not have unit norm, but does describe the correct rotation.
We can projectivise it by using the half tangent formulas for the trigonometric functions. Since this is all in a projective space, the motion polynomial can be multiplied by any scalar - in this case its norm polynomial - and still produce the same motion.
\begin{equation}
    C(u,t) = 1+u^{2}-2tui - t(u^{2}-1)j
\end{equation}
This is the polynomial who's properties are investigated in the remainder of this report. 
\section{Results}
In order to better visualise the procedure being followed, a network will be used to describe the sequence of transformations from one link to another. This is similar to how voltages are described in an electrical network.
First the overall transformation can be described as such:

\begin{figure}[h!]
\begin{center}
\begin{tikzpicture}[block/.style = {draw, fill=white, rectangle, minimum height=3em, minimum width=3em},
sum/.style= {draw, fill=white, circle, node distance=1cm},
input/.style = {coordinate},
tmp/.style = {coordinate},
output/.style= {coordinate},
pinstyle/.style = {pin edge={to-,thin,black}},
auto,
node distance = 2cm,
>=latex'
]


    \node [block, fill=gray!30] (id) {id};
    \node [block, right of=id, fill=green!30, node distance=4cm] (end) {Goal};

    \draw [->] (id) -- node{$C(u,t)$}(end);

\end{tikzpicture}
\caption{The origin is colored gray, and effector transformation is colored green.}
\label{}
\end{center}
\end{figure}
The transformation between 2 vertices is given by the previously described map \ref{act}(this map also works for quaternions). By factorising, we will obtain an equivalent transformation sequence, which is directly realisable mechanically using elementary joints. This can be symbolically described as follows:

\begin{figure}[h!]
\begin{center}
\begin{tikzpicture}[block/.style = {draw, fill=white, rectangle, minimum height=3em, minimum width=3em, node distance=2.5cm},
sum/.style= {draw, fill=white, circle, node distance=1cm},
input/.style = {coordinate},
tmp/.style = {coordinate},
output/.style= {coordinate},
pinstyle/.style = {pin edge={to-,thin,black}},
auto,
node distance = 2cm,
>=latex'
]


    \node [block, fill=gray!30] (id) {id};
    \node [block,right of=id] (link1) {link1};
    \node [block,right of=link1] (link2) {link1};
    \node [block, right of=link2, fill=green!30, node distance=2.5cm] (end) {Goal};

    \draw [->] (id) -- node{$H_1(u,t)$}(link1);
    \draw [->] (link1) -- node{$H_2(u,t)$}(link2);
    \draw [->] (link2) -- node{$H_3(u,t)$}(end);

    \draw [->] (id) to [out=-90,in=-90] node{$C(u,t)$}(end);

\end{tikzpicture}
\caption{The 2 paths to the goal would describe the same motion by construction}
\label{}
\end{center}
\end{figure}

The factorization methods for multivariate biquaterionic polynomials are still in their infancy\cite{ lercher2021multiplication, Lercher_2022 }. The methods used for the univariate case are more developed, but break down in the multivariate case quite spectacularly, producing factors which change their geometry during movement. However, as this is a very simple case, the factorization can be done by hand manually producing the following factorization:
\begin{equation}
    C(u,t) = (1+uk)(1+jt)(1-uk)
\end{equation}
This factorization is interesting, as it can be thought of as a rotation around the y axis, which is first rotated around the z axis. 
The first problem here is that this is a 3R chain which is supposed to perform a 2DOF motion. For that to be the case the movement has to be constrained. In the univariate case this is done by connecting multiple factorizations of the same movement  to the end effector. If described using a diagram it would look like this:

\begin{figure}[h!]
\begin{center}
\begin{tikzpicture}[block/.style = {draw, fill=white, rectangle, minimum height=3em, minimum width=3em, node distance=2.5cm},
sum/.style= {draw, fill=white, circle, node distance=1cm},
input/.style = {coordinate},
tmp/.style = {coordinate},
output/.style= {coordinate},
pinstyle/.style = {pin edge={to-,thin,black}},
auto,
node distance = 2cm,
>=latex'
]


    \node [block, fill=gray!30] (id) {id};
    \node [block,below right of=id] (link1) {link 1};
    \node [block,right of=link1] (link2) {link 2};
    \node [block,above right of=id] (link3) {link 3};
    \node [block,right of=link3] (link4) {link 4};
    \node [block, above right of=link2, fill=green!30, node distance=2.5cm] (end) {Goal};

    \draw [->] (id) -- node{$H_1(u,t)$}(link1);
    \draw [->] (link1) -- node{$H_2(u,t)$}(link2);
    \draw [->] (link2) -- node{$H_3(u,t)$}(end);

    \draw [->] (id) -- node{$G_1(u,t)$}(link3);
    \draw [->] (link3) -- node{$G_2(u,t)$}(link4);
    \draw [->] (link4) -- node{$G_3(u,t)$}(end);

    \draw [->] (id) -- node{$C(u,t)$}(end);

\end{tikzpicture}
\caption{A hypothetical alternate factorization}
\label{}
\end{center}
\end{figure}



Here however that is impossible, as it was found that no other factorization exists. Not to mention, even if it were possible, another constraint would have to be imposed on the mechanism in the form of another loop. This is because according to the mobility formula for spherical mechanisms:
\begin{equation}
    M = J - 3L
\end{equation}
Where J is the number of joints of the mechanism, and L the number of independent loop closure conditions, the mobility of such mechanism would be 3, not 2. Usually these kinds of parallel mechanisms have 3 joints in one arm, and 2 in the other. This way only one loop is needed. It is unclear at the moment how such a factorization could be created, but it would necessitate a different form of the factor representing elementary motions. \\
Another approach was tried, where an artificial scaffolding would be constructed around the main mechanism, which would create the necessary constraints. There are 2 possible topologies for such a scaffold:
\begin{figure}[h!]
\begin{center}
\begin{tikzpicture}[block/.style = {draw, fill=white, rectangle, minimum height=3em, minimum width=3em, node distance=3cm},
sum/.style= {draw, fill=white, circle, node distance=1cm},
input/.style = {coordinate},
tmp/.style = {coordinate},
output/.style= {coordinate},
pinstyle/.style = {pin edge={to-,thin,black}},
auto,
node distance = 2cm,
>=latex'
]


    \node [block, fill=gray!30] (id) {id};
    \node [block,right of=id] (link1) {link1};
    \node [block,right of=link1] (link2) {link2};
    \node [block, right of=link2, fill=green!30, node distance=3cm] (end) {Goal};

    \node [block, above of=id] (ids) {scaf 1};
    \node [block,right of=ids] (link1s) {scaf 2};
    \node [block,right of=link1s] (link2s) {scaf 3};
    \node [block, right of=link2s, node distance=3cm] (ends) {scaf 4};




    \draw [->] (id) -- node{$P_1(u,t)$}(ids);
    \draw [->] (link2) -- node{$P_2(u,t)$}(link2s);
    \draw [->] (end) -- node{$P_3(u,t)$}(ends);

    \draw [->] (id) -- node{$H_1 = 1+uk$}(link1);
    \draw [->] (link1) -- node{$H_2 = 1+jt$}(link2);
    \draw [->] (link2) -- node{$H_3 = 1-uk$}(end);

    \draw [->] (ids) -- node{$G_1(u,t)$}(link1s);
    \draw [->] (link1s) -- node{$G_2(u,t)$}(link2s);
    \draw [->] (link2s) -- node{$G_3(u,t)$}(ends);

    \draw [->] (id) to [out=-90,in=-90] node{$C(u,t)$}(end);

\end{tikzpicture}
\begin{tikzpicture}[block/.style = {draw, fill=white, rectangle, minimum height=3em, minimum width=3em, node distance=3cm},
sum/.style= {draw, fill=white, circle, node distance=1cm},
input/.style = {coordinate},
tmp/.style = {coordinate},
output/.style= {coordinate},
pinstyle/.style = {pin edge={to-,thin,black}},
auto,
node distance = 2cm,
>=latex'
]


    \node [block, fill=gray!30] (id) {id};
    \node [block,right of=id] (link1) {link1};
    \node [block,right of=link1] (link2) {link2};
    \node [block, right of=link2, fill=green!30, node distance=3cm] (end) {Goal};

    \node [block, above of=id] (ids) {scaf 1};
    \node [block,right of=ids] (link1s) {scaf 2};
    \node [block,right of=link1s] (link2s) {scaf 3};
    \node [block, right of=link2s, node distance=3cm] (ends) {scaf 4};




    \draw [->] (id) -- node{$P_1(u,t)$}(ids);
    \draw [->] (link1) -- node{$P_2(u,t)$}(link1s);
    \draw [->] (end) -- node{$P_3(u,t)$}(ends);

    \draw [->] (id) -- node{$H_1 = 1+uk$}(link1);
    \draw [->] (link1) -- node{$H_2 = 1+jt$}(link2);
    \draw [->] (link2) -- node{$H_3 = 1-uk$}(end);

    \draw [->] (ids) -- node{$G_1(u,t)$}(link1s);
    \draw [->] (link1s) -- node{$G_2(u,t)$}(link2s);
    \draw [->] (link2s) -- node{$G_3(u,t)$}(ends);

    \draw [->] (id) to [out=-90,in=-90] node{$C(u,t)$}(end);

\end{tikzpicture}
\caption{Second scaffold creates a 4 bar near the base of the mechanism, whle the first near the end}
\label{}
\end{center}
\end{figure}

\clearpage
The polynomial $P_2$ can be more or less chosen arbitrarily. It was chosen however, to make the 4 bars univariate. In each of these topologies, there are 3 ways to choose the factors, giving 6 cases to consider. Regrettably, none of these cases produced a viable mechanism. 
\\
Finally, to determine where it all went wrong, an attempt was made to analyse an actual 2DOF spherical manipulator using this method. The manipulator chosen was a simplified version of the 3DOF "Agile Eye" manipulator developed by dr. Gosselin . 
A system of equations was developed based on the description of how the passive joint angles depend on the 2 active ones. It was found that its solution was impossibly complicated, included multiple nested square roots, and was in a singular configuration at the origin. This could have several explanations, the most probable of which is that the assumption of 2 factorizations, one having 3 linear factors, the other 2, was wrong. It remains to be seen what would be the appropriate way to describe such a system with quaterionic polynomials.

\section{Discussion}
On the surface the project can be deemed a failure. The goal of creating a manipulator using polynomial factorization was not met. However where the 
issues with this method lie was greatly elucidated. Decomposition into
linear factors is something that mathematicians are interested in, but 
for the purpose of manipulator design, it may be more worthwhile to look
into factorizations int o factors producing linear, or rotational motion. 
For example, one way to represent a translation using a biquaterionic polynomial is:
\begin{equation}
    H(y) = 1 - \varepsilon t v
\end{equation}
However, another way to do it is with a quadratic polynomial:
\begin{equation}
    H(t) = 1 +t^2 + t^{2} \varepsilon v
\end{equation}
This is still a linear motion, but bounded instead of unbounded. 
Similar alternative representations can be found for rotational motion.
Future investigations should attempt to use these alternative representations in the multivariate case to construct viable mechanisms.

%\nocite{*}
\printbibliography %Prints bibliography

\end{document}
