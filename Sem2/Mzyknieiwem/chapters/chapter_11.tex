
\chapter{Parametric estimation of non-linearities in a Hammerstein system}


If $\mu(\cdot) = a_1f_1(\cdot) + \dots + a_mf_m(\cdots)$, then the output is linear
with respect to the parameters, no mater what $f_i$ are.
\nt{In the literature, the Hammerstein system is considered a solved problem. What the engineers are now trying to accomplish with respect to it
is to simply increase the rate of convergence}
\chapter{Wiener system}

A Wiener system is essentially an inverted Hammerstein system. In a Wiener system, the dynamics are applied before the non-linearities.
The problem can be stated as follows:
\qs{Wiener system identification}
{
    Using the set of tuples $\left\{ (u_k,y_k) \right\}_{k=1}^{N}$, estimate both $\mu(\dot) $ and  $\gamma_i$.
}
\ex{}
{
Let $\mu(x) = x^{2}$, and $\{\gamma_i\} = 1, 1, 0, 0, 0, \dots$.
Then we have:
\begin{equation}
    x_k = 1u_k  1u_{k-1}
\end{equation}
Which gives:
\begin{equation}
    y_k = (u_k - u_{k-1})^{2} + z_k = u_k^{2} + u_{k-1}^{2} + 2u_ku_{k-1} + z_k
\end{equation}
Even in this simple system, we already have a problem.
}

\section{Vector norms}
I know this
\section{Aproximaton)}

Given:
\begin{equation}
    u(\gamma^{T},\phi_k),\; \gamma =  \begin{bmatrix}
        \gamma_0 \\
        \vdots\\
        \gamma_S
    \end{bmatrix}  \\
    ,\; \phi_k = \begin{bmatrix}
        u_k  \\
        u_{k-1}\\
        \vdots\\
        \u_{k-S}
    \end{bmatrix}
\end{equation}

Tjem we can obsrve:
\begin{equation}
    (\mu(cx) )'= c\mu'(cx)
\end{equation}
If we can guarantee $\norm{\phi_k} \le R \rightarrow x_k \le c_2R$ 

We can estimate a wiener system using the kernel functions, as for small $h$ they parametrize liear approximations f our both desiered fnctio ,\

as follows:
\begin{equation}
    \hat{\gamma} = \left( \Sigma_{k=1}^{N}\phi_k\phi_K^{T}K(\frac{\norm{\phi_k}}{h}  )\right)^{-1}\Sigma\phi_k y_k K(\norm{\phi_k}/h)
\end{equation}

The norm can be apbitrary
Which givews a parametric non parametric approach to aprametrization.

\section{Asymptoti propertiers}

 \begin{itemize}
        \item $h \rightarrow  0$
        \item Probabilityt of selection $P(K(\frac{\norm{\phi_k}}{h}) = 1) \sim h^{s+1} \leftarrow$ Expected no of selected situations is $Nh^{S+1}$\\
            This condition gives $Nh^{s+1 \rightarrow \infty}$
\end{itemize}




