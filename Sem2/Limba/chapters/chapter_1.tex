\chapter{Motivation}
\section{Parallel mechanisms}

\dfn{Parallel Kinematic Chain}
{
    A purely parallel kinematic chain is a mechanism, in which multiple serial chains are connected to the end effector.
    For the sake of this presentation however, hybrid manipulators will also be considered parallel.\\
    A hybrid manipulator is any kinematic chain containing closed transformation loops.\\
    The simplest example of a parallel mechanism is the planar 4-bar:
    %img
    \\
    However more complex structures are also commonly employed in robotics:
    %
    %delta i gough zdjecia
}
\subsection{Characteristics}
\begin{itemize}
        \item Mass concentrated near the base
        \item Higher stability
        \item Complex workspaces
        \item Difficult mathematical description
\end{itemize}

\subsection{Analysis}
\begin{itemize}
        \item Mostly vector based
        \item Often exploit symmetry
        \item Analytic solutions to problems are rare
        \item More types of singularities
\end{itemize}

\subsection{Synthesis}
\begin{itemize}
        \item Mostly based on ingenuity
        \item Systematic methods are still trapped in the serial-chain framework
        \item Often exploit Lie sub algebras of the displacement group
        \item Generally lacking
\end{itemize}
\subsubsection{Virtual chain method}
\dfn{Virtual chain method}
{
    The Virtual chain method was developed by dr. Kong and dr. Gosselin for the description of the desired motion of the end effector of the synthesised manipulator.\\
    It worked by describing a serial manipulator following a desired motion, and then constructing parallel legs of an appropriate type to restrict the motion of this virtual manipulator. This was done using Lie Algebraic methods, by creating a catalogue of serial chains, along with the wrenches constricting them.
    The intersection of these constraints should at the end produce the desired motion.
}


\section{Deficiencies}

The Virtual chain method suffers from the clear issue of needing an a priori description of the desired motion, without having a well described apparatus to that purpose. 
Even while describing their method, they described 2 types of motion which could not be realised by serial chains, but gave no systematic way to arrive at them. This ultimately makes the task of synthesising a parallel mechanism require a large deal of ingenuity, which greatly restricts their use to a few well described examples.



