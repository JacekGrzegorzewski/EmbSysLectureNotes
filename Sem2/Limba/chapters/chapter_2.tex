\chapter{Rigid Body Motion}
In order to talk about manipulator synthesis, one first has to provide a
description of Euclidean displacements. This can be done by considering the Special Euclidean Group in 3 dimensions. A group can be represented in many different ways, and as such by carefully chosing this representation one can approach the group from a different angle.
\section{Group theoretic background}
The Special Euclidean Group(SE(3)) is composed of 2 main subgroups. One
representing rotations, and the other translations in 3D space. These 2 subgroups interact with each other in a not exactly pleasant way.
\subsection{SO(3)}
The special orthogonal group is the subgroup representing rotations around the origin in 3D space.
\subsection{T(3)}
The translational group consists of all translations in three dimensions.
\subsection{SE(3)}
SE(3) can be described as a semi-direct product of T(3) and SO(3):
\begin{equation}
    SE(3) = SO(3)\ltimes T(3)
\end{equation}
With this construction, elements of SE(3) are given by a tuple  $(r,t)$, and a product of 2 of its elements is given as follows:
 \begin{equation}
     \label{semidirect}
     (r_2,t_2)(r_1,t_1) = (r_2r_1,r_2t_1\bar{r}_2+t_2)
\end{equation}

\section{SE(3) Representation}
The way roboticists usually interact with SE(3) is through uniform
transformation matrices. The goal of constructing these matrices was to create a compact representation of the group, which would use only one operation.
What will be demonstrated, is how one can arrive at these constructions starting from regular rotation matrices and vectors, and how a similar proes leads to biquaternions from quaternions. 
\subsection{Rotation Matrices and Quaternions}

\subsection{Uniform Matrices and Biquaternions}
\subsection{Comparison}










