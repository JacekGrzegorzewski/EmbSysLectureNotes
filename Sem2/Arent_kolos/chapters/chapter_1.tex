\chapter{Something}
\section{Basic feedback}

A general feedback system is composed of:
\begin{itemize}
        \item r - reference signal
        \item $\phi$ - input output equation of the plant
        \item y - output of the whole system
        \item  $\psi$ - I/O of the feedback compensator
        \item e - error
        
\end{itemize}
By composing $\gama = \psi \dot \phi$ we get what's called a loop IO map, from which we obtain a new
equation called the feedback equation.


\\
The feedback of the loop IO map is most effective if  $\gamma$ has large gain.
When that is the case, we can be sure that $\gamma(e) \approx r$, which is close to the idol case.
This is under the assumption that e is bounded.

\\
The robustness of the system states that despite disturbances we have  $\gamma(e) \approx r$.
By rewriting the original system subject to disturbances d, we get a different system:
 \begin{equation}
    z = d - \delta(z)
\end{equation}
Here again if the gain is large, we say that the output z is insensitive to the disturbance d.
\\
This high gain property  of $\delta$ decreases the gain of the whole system, but makes the linear
bad much wider. This is essentially the origin of control, as the concept of a feedback loop for
an electronic vacuum tube system was stated in a patent as a way to extend the linear band of the
system. 

\section{Robustness of feedback}
Robustness is defined as the system remaining stable under changes to the plant and compensator.

Robustness of feedback deals with some properties of the Nyquist plot of a known, for the moment open-loop stable, system which
has been closed. We need to keep the Nyquist plot as far away from  the point -1 as possible to ensure good behaviour of the system.
\\
The classical approach used 2 parameters to indicate this closeness:
 \begin{itemize}
        \item $\phi_m$ - phase margin
        \item  $k_m$ - gain margin
\end{itemize}
These however were found to be insufficient to fully describe robustness, so 2 more were considered:
\begin{itemize}
        \item $s_m$ - modolus margin
        \item  $\tau_m$ - delay margin
\end{itemize}

The classical approach establishes robustness by constraining minimum values of the gain and phase margin.\\
Adequate values of these margins give not only robustness, but also satisfactory response characteristics.
This is because if the Nyquist plot is close to -1, we get zeros close to the imaginary axis.

 
\subsection{Open gain}

The above specifications brake down for open-loop unstable systems and MIMO systems, as such,
new criterion's are needed. One such criterion is the Doyle criterion. Say the system is perturbed
from the nominal loop gain $L_0$ to the actual loop gain $L$. Doyle's criterion states, that the system 
remains stable under perturbations not affecting the number of open-loop positive poles if:
 \begin{equation}
     \frac{ \norm{L - L_0}}{\norm{L_0}} < \frac{1}{\norm{T_0}}
\end{equation}
Where $T_0$ is the nominal complimentary sensitivity of the closed loop system:
\begin{equation}
    $T_0 = \frac{L_0}{1+L_0}$
\end{equation}
This is merely a sufficient condition.

There is another criterion  called the inverse loop stability, which is exactly the same as
the Doyle one but deals with inverses on the left side, and the right side has to be less than $\frac{1}{\norm{S_0}}$.
The relation between $S_0$ and $T_0$ is $S_0 = 1-T_0$.

To make the sytem stable, we need both $S_0$ and $T_0$ to be small, which is impossible since the both add up to 1.
A solution to this is to make them small in different frequency ranges. Usually $S_0$ is small at low frequencies, while
$T_0$ at high frequencies.

\section{Guillemin-Truxal procedure}

In this procedure, we we assume a known plant $P$. The closed loop transfer function then is:
 \begin{equation}
    H = \frac{PC}{1+PC}
\end{equation}

We then choose some desired $H$, and solve for C:
 \begin{equation}
    C = \frac{1}{P} \frac{H}{1-H}
\end{equation}
This compensator should give the desired H. 
Choosing an appropriate H is not a simple take, usually one starts by investigating H for
low order compensator C. More specifically one analyses a closed loop system of type k, where k is the number
of poles at the origin.
\\
Then one still has a lot of free variables to determine for $H$. There are multiple procedures to choose these
coefficients.
Some filters which can be chosen include:
 \begin{itemize}
        \item Butterworth polynomials
        \item Chebychev polynomials
        \item Elliptic(Chebychev) rational funcitons
        \item ITAE standard forms
        \item etc.
\end{itemize}

\section{Adaptice Control}
Mzyk i dupa
\section{Adaptive pole placement}

