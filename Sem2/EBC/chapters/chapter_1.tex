
\chapter{Petri Nets}
Petri nets are bipartite graphs, with the 2 types of vertices being:
\begin{itemize}
        \item Places -- represented by circles and denoted by the letter \textit{S}
        \item Transitions -- represented by bars or boxes
\end{itemize}
No 2 vertices of the same type can connect to each other.\\
\section{Formal definition}
A petrinet is a 5-tuple $N = (P,T,F,W,M_0)$ s.t.:
\begin{itemize}
    \item P -- a finite set of places
    \item T -- a finite set of transitions
    \item $P \cap T = \emptyset$ -- both of these sets are disjoint, a place can't also be a transition and vice versa.
    \item  $F \subset (P\times T)\cup(T\times P)$ -- the flow relation( depicted by arcs)
\end{itemize}
\nt{Notation for the sets of inputs and outputs of a place or transition is as follows:}
    
\ex{Example of a net}
{
        \begin{tikzpicture}
            \node[place,
    label=above:$P_2$,
    label=right:$P_2$,
    label=left:$P_2$,
    label=below:$P_2$] (place2) at (2,0) {};
        \end{tikzpicture}
}

\section{Algebraic representation of P / T net}
Consider the system $N = (P,T,F,W,M_0)$ with the linear ordered set of places and transitions: $P = \{p_1,p_2,\cdots ,p_{\norm{P}}\}$ and $T = \{t_1,t_2,\cdots ,t_{\norm{T}}\}$
\begin{itemize}
    \item input incidence matrix: $X^{-} = [x^{-}_{ij} = W(p_i,t_j)_{\norm{P}\times\norm{T}}]$
    \item output incidence matrix: $X^{+} = [x^{-}_{ij} = W(t_j,p_i)_{\norm{P}\times\norm{T}}]$
    \item incidence matrix: $X^{+}-X^{-}$
\end{itemize}

\subsection{Fundamental equation}
Consider a P / T net $N = (P,T,F,W,M_0)$ at marking M.\\
\dfn{Firing sequence}
{
 A firing sequence is a sequence:
 \begin{equation}
     \sigma = \tau_1,\tau_2,\cdots ,\tau_n
 \end{equation}
 such that $\tau_i \in T$ and $M \equiv M_1\overset{\tau_1}{\rightarrow}M_2 \cdots \overset{\tau_{n-1}}M_n$
}
\dfn{characteristic vector of sequence $\sigma$}
{
    A characteristic vector of a sequence is a column vector:
    \begin{equation}
        Y_{\sigma} = [y_i \; : \; i\in 1 \cdots \norm{T}]
    \end{equation}
    where $y_i$ is the number of occurences of transition  $t_i$ in $\sigma$
}
The above allows for defining state transition by matrix multiplication and vector addition.\\
\dfn{Fundamental equation}
{
    If $M \overset{\sigma}{\rightarrow} M'$ then:
     \begin{equation}
         M' = M + X Y_{\sigma}
    \end{equation}
}


\ex{Add examples}
{...}
