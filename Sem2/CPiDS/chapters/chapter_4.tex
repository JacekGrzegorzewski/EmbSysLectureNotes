
\chapter{Robotic simulation}

\dfn{Simulation}
{
    \begin{itemize}
            \item A simulation imitates the operation of real system or processes with the use of models
            \item the model represents behaviours and characteristics of the selected process or systems
            \item The simulation presents how the model evolves under different conditions over time
    \end{itemize}
    Why are systems simulated:
    \begin{itemize}
            \item Less financial risk
            \item Repeated testing in the same conditions
            \item To generate synthetic data
            \item Examine long-term impacts
    \end{itemize}
}

\section{Introduction}
\begin{itemize}
        \item Robotics Simulation: The process of mimicking the behaviour of a robot in a virtual environment.
        \item Purpose: To model, analyze, and test robot systems
        
\end{itemize}
\section{Applications}
\begin{itemize}
        \item Algorithm development: Testing control algorithms in a safe and controlled environment
        \item Prototyping: Designing and iterating robot models before physical construction
        \item Training: Training machine learning models for perception and decision-making
        \item Verification and Validation: Validating the performance of robotic systems
\end{itemize}

\section{Components}
\begin{itemize}
    \item Robot model: Mathematical representation of the robot.
    \item Sensor models: Rmulation of sensors like cameras, lidars, IMUs, etc.
    \item Algorithms: Perception, planning, control algorithms
    \item Environment: Virtual representation of the robot's surroundings 
    \item Physics engine: Simulation software that calculates the interactions between objects in the virtual world.
\end{itemize}

\subsection{Physics engines}
\begin{itemize}
        \item Simulates physical phenomena in a virtual environment
        \item Used for realistic simulations in gaming, engineering, and animation
        \item Components:
            \begin{itemize}
                    \item Rigid body dynamics
                    \item collision detection
                    \item Integration methods
                    
            \end{itemize}
        \item Popular engines:
            \begin{itemize}
                    \item Bullet physics
                    \item Nvidia PhysX
                    \item ODE(Open Dynamics Engine)

                    
            \end{itemize}

        \item Challages: Overcoming computational complexity and accuracy issues
        \item Future Developments: Trends include improved real-time simulations and AI integration.
\end{itemize}


\subsection{Popular simulation tools}
\begin{itemize}
        \item Gazebo: An open-source simulation tool widely used in the robotics community
        \item Nvidia Isaac Sim: A simulation environment focused on the physics simulation and acceleration based on GPU computation
            \oitem O3DE(Open 3D engine): General purpouse robotics simulator
        \item V-REP(CoppeliaSim): Versatile robot eexperimentation platform with extensive features
        \item Webots: A development environment used for modeling, simulaiton, a control of mobile robots
        \item CARLA: Mostly for vehicles
        \item Unity
        
\end{itemize}

\section{Gazebo}
\begin{itemize}
        \item Introduction:
            \begin{itemize}
                    \item Gazebo is a nopen-source robotics simulation software
                    \item Developed by the open Source robotics foundation
                    
            \end{itemize}
        \item Features:
            \begin{itemize}
                    \item Dynamics simulation: works with multiple high-performance physics engines.
                    \item 3D visualization for modeling and testing robot designs.
                    \item Extensive library of sensors (camers, lidars, RGB-D sensors, IMU,GPS, etc.), actuators models, and environments
                    \item TCP/IP integration
                    
            \end{itemize}
        
\end{itemize}

\section{URDF}
\dfn{URDF}
{
    URDF (Unified Robot Description Fromat) is a standard XML format for describing a robot's structure , kinematics and dynamics.
    Its purpouse is modeling and visualization of robots, aiding in simulation and control
}

Features:
\begin{itemize}
        \item Standardization: Provides a standardzed format for robot descriptions.
        \item Interoperability: Enables interoperability among different ROS packages and tools.
        \item Simulation: Facilitates accurate simulation of obot behaviours
        
\end{itemize}

\subsection{Components}
\begin{itemize}
        \item Link: rigid bodies comprising the kinematic chains
        \item Joint: Defines the connection between two links and their relative motion
        \item Material: Specifies visual and collision properties
        \item Transmission: Describes how joint efforts result in link movements
        
\end{itemize}





\section{ODE}

\dfn{ODE}
{
    An Ordinary Differential Equation(ODE) is an equation that involves one or more derivatives of an unknown function with respect to one independent variable.
}

\ex{}
{
    A general form of a firts-order ODE is:
    \begin{equation}
        \frac{dy}{dx} = f(x,y)
    \end{equation}
}

VERY IMPORTANTTT!!!@!!111!!11!


\ex{UUV}
{

    \begin{equation}
        M\dot{v} + C(v)v + D(v)v + g(\eta) = \tau
    \end{equation}
    Where:
    \begin{itemize}
        \item M: inertia matrix\\
        \item $C(v)$ : Coriolis and centripetal terms
        \item $D(v)$ : damping matrix\\
        \item g(\eta) :vector of grabvitationalf o.....
            jebavc kurwa mac jebana
    \end{itemize}

}
\subsection{ODE solving with SciPy}
Features:
\begin{itemize}
    \item Open-Source library for mathematics, science and engineering
    \item ... gowno
\end{itemize}


\section{Optimization}

\begin{itemize}
        \item Types:
            \begin{itemize}
                    \item Unconstrained
                    \item Constrained
                    \item Global
                    \item Local

                    
            \end{itemize}
        \item Algorithms:
            \begin{itemize}
                    \item Gradient Descent
                    \item Genetic Algorithms
                    \item Simulated annealing
                    \item Particle Swarm
                    
            \end{itemize}
        \item Applications:
            \begin{itemize}
                    \item ML
                    \item Operations Research
                    \item Engineering
                    \item Finance

                    
            \end{itemize}
        \item Challenges: Non-Convexity, High-dimensionality
       
        
\end{itemize}


Popular optimization tools:
\begin{itemize}
        \item Ceres Solver
        \item g2o
        \item CasADI
        \item $\cdots $
\end{itemize}


\subsection{Ceres Solver}

\begin{itemize}
        \item Introduction:
            \begin{itemize}
                    \item Ceres Solcer is an open-source C++ library for modeling and solving large, coplex optimization problems
                    \item Developed by google and named after the dwarf planet Ceres???? What
                    
            \end{itemize}
        \item Features:...

\end{itemize}
who cares....



\section{Linear algebra}

\dfn{}
{
    \begin{itemize}
            \item C++ template library for linear algebra
            \item Designed for high-performance matrix and vector calculation
            
    \end{itemize}
}
Key features:
\begin{itemize}
        \item Expressiveness: Intuitive syntax for linear algebra operations
        \item Performance: Emphasis on efficient and fast computations
        \item ...
        
\end{itemize}

...









