\chapter{Equations of motion}
\section{Introduction}
The dynamics of an aerodynamically controlled rocket are very complex, and mostly of empirical nature.
The main source of problems in their analysis are the aerodynamic force and moment coefficients. Their
vales depend not only on the geometry of the rocket, but also on its orientation relative to the velocity
vector, as well as deflection angles of the control surfaces in highly non-linear ways. By performing
CFD simulations, one can obtain input output pairs for a given state, and use those to estimate the coefficients
at those points. If, additionally, one possesses a set of specific equations describing the proverbial functional shape of these coefficients, regression can be used to fit them more accurately to their form.
Below is a possible list of models for aerodynamic forces and moments, along with their advantages and disadvantages:
\section{Aerodynamic Forces}
\subsection{The general model}
The general model of aerodynamic forces acting on a missile is usually written as:
\begin{equation}
    \begin{cases}
        
        F_B = \frac{1}{2}\rho S_{\text{ref}}v_B^{2}\mathbf{C}^{B}_A \begin{bmatrix}
            -C_D \\
            C_S\\
            -C_L
        \end{bmatrix}
        \\
        M_B = \frac{1}{2}\rho S_{\text{ref}}l_{\text{ref}}v_B^{2}\mathbf{C}^{B}_A \begin{bmatrix}
            C_l \\
            C_m\\
            C_n
        \end{bmatrix}

    \end{cases}
\end{equation}
Where $\mathbf{C}^{B}_A$ is the transformation matrix from the chosen aerodynamic frame to the body frame. \\
The main difficulty with these equations is that each of these coefficients $C_i$ is highly non linear, and depends not only on the mach number and orientation of the body relative to the velocity vector, but also on the deflection angles, in our case  $C_i(M,\alpha,\beta,\delta_1,\delta_2)$. This, along with not tetragonal but planar symmetry, necessitates performing a large number of simulations, from which we will obtain a 4 dimensional look-up table(Only one mach number will be considered). This is very unwieldy in theoretical applications, and as such several alternatives will be considered.
\nt{Some sources don't transform the moments to the aerodynamic frame, i think that's a mistake, but will have to look further into it}

\subsection{Summed model}
This model is described by the equations below:

\begin{equation}
    \begin{cases}
        F_B = \frac{1}{2}\rho S_{\text{ref}}v_B^{2}\mathbf{C}^{B}_A \begin{bmatrix}
            -C_D \\
            C_S\\
            -C_L
        \end{bmatrix}
        \\
        M_B = \frac{1}{2}\rho S_{\text{ref}}l_{\text{ref}}v_B^{2}\mathbf{C}^{B}_A \begin{bmatrix}
            C_l \\
            C_m\\
            C_n
        \end{bmatrix}

}v_B^{2} \left( \mathbf{C}^{B}_A
    \end{cases}
\end{equation}

