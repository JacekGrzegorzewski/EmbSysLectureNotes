
   \ex{}
   {
       Assume tthat we have processes $P_1$, $P_2$,$\cdots $, $P_n$. Sometimes, they use common resources:
               $q_i$ --  $P_i$ units have access to resources,
               $r_i$ --  $P_i$ uses resources.
               
               \begin{itemize}
                       \item We want to avoid a situation, where 2 processes use common resources at the same times:
                           \begin{equation}
                               \Box \bigvee_i (r_i \rightarrow \bigvee_{j \neq i} \neg r_j) \text{-- There is no conflict}
                           \end{equation}
                       \item none will be starved \\
                           \begin{equation}
                               \bigvee_i \Box (q_i \rightarrow q_i \mathcal{U}r_i)
                           \end{equation}
                       \item   \begin{equation}
                                  \bigvee_i \Box (r_i \rightarrow \Diamond \neg r_i)
                              \end{equation} 
               \end{itemize}
   }
%
%   \ex{}
%   {
%     $\Diamond p$ means p will be true at least once $\Diamond (p \wedge \Circle \Diamond p)$ means that p will be true at least twice.
%   }
%
   \ex{}
   {
       $p \wedge  \Diamond (p \rightarrow \Circle \neg p) \wedge \Box (\neg p \rightarrow \Circle p)$ 
       means that pi is true in even moments, and false in odd moments
   }

\ex{Properties not describable in LTL}
   {
       "p is true in even points"
       Formally: $phi \in \mathcal{L}_{LTL}(\mathcal{P})$\\
       $mod( \phi ) = \{ \;\pi : \mathbb{N} \rightarrow P(\mathcal{P}) \mid \pi , 0 \vDash \phi \,\}$ \\
       $mod(\Diamond p) = \{ \;\pi : \mathbb{N}\rightarrow P(\{p\}) \mid \exists n \pi (n) = \{p\}  \,\}$
       \nt{There is no $\phi \in \mathcal{L}_{LTL}(\{p\})$ such that $mod(\phi ) = \{ \;\pi :(\forall n )(\pi (2n) = \{p\})$
   }
} 
}
\thm{Wolper}
{
    Let $\rho_i = \{0,0,\cdots ,{p},0,0,\cdots \}$ or equivalently:\\
    \begin{equation}
        \rho_i(j) =
        \begin{cases}
          \case  0 : j \neq i
    \case \{p\} : j = i
        \end{cases}
    \end{equation}
}

Let $\phi  \in \mathcal{L}_{LTL}(\mathcal{P})$. If i,j > nd($\phi $), then
$\rho_i, 0 \vDash \phi \leftrightarrow \rho_j, 0 \vDash \phi $.
\nt
{
    nd: $\mathcal{L}_{LTL}(\mathcal{P}) \rightarrow \mathbb{N}$ is defined by recursion:
    \begin{enumerate}
        \item $nd(p) = 0$
        \item $nd(\phi \vee \psi ) = \max\{nd(\phi ),nd(\psi )\}$
        \item $nd(\neg \phi ) = nd(\phi)$
        \item $nd(\Circle \phi ) = nd(\phi ) + 1$
        \item $nd(\phi \mathcal{U} \psi ) = \max\{nd(\phi ),nd(\psi )\}$

    \end{enumerate}
}

\nt
{
    Wolper's theorem implies the fact, that:
    Assume taht there is $\phi $ satisfying fact.
    Then we can find $\phi $ s.t. \\
    $mod(\phi ') = \{ \pi : (\forall n)(\pi (2n) = \null  \}$ 
    It is enough to substitute p by $\neg p$.

    \ex{Wolper's example}
    {
        $\rho_i \ in mod(\phi ') \equiv 2 \not \mid i$
        Take $2n, 2n+1 > nd(\phi ')$
        $\rho_{2n+1}, 0 \vDash \phi ' \leftrightarrow \rho_{2n+1}, 0 \vDash \phi '$, a contradiction.
    }
}


\dfn{Finite Automata}
{
    A is NFA(non-deterministic finite automaton) if: $\mathbb{A} = (S,S_0,\delta,F,\Sigma)$, where:
    \begin{itemize}
            \item S - a finite set (a set of states)
            \item $S_0 \subset S$ (a set of inital states)
            \item $F \subset S$ (a set of accepted states)
           \item $\Sigma$ - a finite set(an alphabet)
            \item  $\delta\,:\,S\times\Sigma \longrightarrow P(S)  $ (a transistion function)

            
    \end{itemize}
}

\ex{}
{
    Let:
    \begin{itemize}
        \item $\Sigma = \{0,1\}$
        \item S = {a,b,c}
        \item  $S_0$ = {a,b}
        \item F = {a,c}
            
    \end{itemize}
    \begin{center}
    \begin{tabular}{|c|c|c|}
    \hline
    S & $\Sigma$ &  $\delta$  \\
    \hline
    a & 0 & {a,b} \\
    \hline
    a &1 &{a}\\
    \hline
    b & 0 & {c} \\
    \hline
    b & 1 & \null \\
    \hline
    c & 0 & {b}\\
    \hline
    c & 1 & {b,c}\\
    \hline
    \end{tabular}
    \end{center}
HERE BE PICTURE

}


\dfn{DFA}
{
    $\mathbb{A}$ is DFA(Deterministi finite automaton) if
    $\mathbb{A}$ is NFA and $(\forall s \in S )(\forall a \in \Sigma ) |\delta(s,a)| = 1$ and $|S_0| = 1$.
    \nt
    {
        Usualy we define DFA not n terms of NFA, but using a similar definition
    }
}

\dfn{Language}
{
    Let $\mathbb{A}$ be NFA. $L(\mathbb{A}) \subset \Sigma^*$ is a language accepted by $\mathbb{A}$. Where:
    \begin{itemize}
        \item $\Sigma^* = \bigcup_{n\in \mathbb{N}} \Sigma^n$ -- set of all words over  $\Sigma$
        \item $\Sigma^n$ -- set of words of length n
            
    \end{itemize}
 Furthermore:\\
 $(a_0,a_1,\cdots ,a_{n-1}) \in L(\mathbb{A})$ if and only of $\exists (t_0,t_1,\cdots ,t_n) \in S^{n+1}$ s.t.
 \begin{itemize}
         \item $t_0 \in S_0$
         \item $\forall_i t_{i+1} \in \delta(t_i,a_i) $ 
         \item $t_n \in F$
         
 \end{itemize}
 \ex{}
 {
     $\mathbb{A}$ = 
     HERE BE PICTURE
     $L(\mathbb{A}) = 0(0+1)*$ = all words started with 0(this is just a regexp)
 }
 \ex{}
 {
    $\mathbb{B}$ = 
    ANOTHER PICTURE
    $\epsilon$ is an empty word
    $\mathbb{B} = \epsilon + 0(0+1)*1$
 }
}

\thm{}
{
    Assume that $\mathbb{A}$ is NFA. Then there exist $\mathbb{B}$ DFA s.t.
    $L(\mathbb{A}) = L(\mathbb{B})$
}

\begin{myproof}
    Or rather a sketch of one \\
    $\mathbb{A} = (S,S_0,\delta,F,\Sigma)$ \\
    We will construct $\mathbb{B} = \{P(S),S_0,\bar{\delta},\bar{F},\Sigma\}$ \\
    $\bar{F}= \{T \subset S: T \cap F \neq \null\}$\\
    $\bar{\delta}(X,a) = \bigcup_{x\in X}\delta(x,a)$

    HERE BE PICTURE
\end{myproof}


\thm{}
{
    Let $\mathbb{A}, \mathbb{B}$ be NFA. Then there is $\mathbb{A}\oplus \mathbb{B}$ NFA such that $L(\mathbb{A})\cup L(\mathbb{B}) = L(\mathbb{A}\oplus \mathbb{B})$.
\nt
{
    The resulting automaton is nondeterministic, even if $\mathbb{A}, \mathbb{B}$ are deterministic.
}
\qs{}
{
    Try to find "nice" construction in the case $\mathbb{A}, \mathbb{B}$ DFA
}
}

\begin{myproof}
    Let:
    $\mathbb{A} = (S_A,S_{OA},\Sigma,\delta_A,F_A)$ \\
    $\mathbb{B} = (S_B,S_{OB},\Sigma,\delta_B,F_B)$ \\
    we can assume that $S_A \cap S_B = \empty$\\
    $\mathbb{A}\oplus \mathbb{B} = (S_A\cup S_B,S_{OA}\cup S_{OB},\cdots ,F_A \cup , F_B)$\\
   $ 
        \delta_A \cup \delta_B(s,a) = \begin{cases}
            \delta_A(s,a) \text{ if } s\in S_A\\
            \delta_B(s,a) \text{ if } s\in S_B
        \end{cases}
    $ 

\end{myproof}

\thm{}
{
    If $\mathbb{A}, \mathbb{B}$ are NFA then there is $\mathbb{A} \otimes \mathbb{B}$ NFA such that:\\
$L(\mathbb{A} \otimes \mathbb{B}) = L(\mathbb{A}) \cap L(\mathbb{B})$\\

    \begin{myproof}
        $\mathbb{A} \otimes \mathbb{B} = (S_A \times S_B,\cdots ,F_A \times F_B )$\\
$ \delta((s,t),a) = \delta_A(s,a)\times \delta_B(t,a)$
    \end{myproof}
    And a nicer proof:
    \begin{myproof}
        Let us prove that:\\
         $L(\mathbb{A}) \cup L(\mathbb{B}) \subset L(\mathbb{A}\otimes \mathbb{B})$\\
         Let $w \in \Sigma^{*},\; w \in L(\mathbb{A})\in L(\mathbb{B})$ \\
         $w = (a_0,a_1,\cdots ,a_n)$\\
         There is a $\mathbb{A} - \text{run}(s_0,s_1,\cdots ,s_{n+1}) \in S_A^{*}$\\
         $s_0 \in S_{OA}, (\forall i ) s_{i+1}\in\delta_a(s_i,a_i), s_{n+!} \in F_A$\\
         There is a $\mathbb{B} - \text{run}(t_0,t_1,\cdots ,t_{n+1}) \in S_B^{*}$\\
         $t_0 \in S_{OB}, (\forall i ) t_{i+1}\in\delta_a(t_i,a_i), t_{n+!} \in F_B$
    \end{myproof}
    \nt
    {
        Notice that $((s_0,t_0),\cdots ,(s_{n+1},t_{n+1}))\in(S_A \times S_B)^{*}$ is a $(\mathbb{A} \otimes \mathbb{B})$ -- run and\\
        $(s_0,t_0) \in S_{OA} \times S_{OB}$ and $(\forall i)((s_{i+1},t_{i+1}) \in \delta((s_i,t_i),a_i)) $ and $(s_{n+1},t_{n+1}) \in F_A \times F_B$
    }
}

\nt
{
    If $\mathbb{A}, \mathbb{B}$ are DFA, then $\mathbb{A}\otimes \mathbb{B}$ is also DFA
}

\thm{}
{
    Assume $ \mathbb{A}$ is DFA. Then there is DFA $\mathbb{A}^{c}$ such that:\\
    $L(\mathbb{A}^{c}) = \Sigma^{*}\backslash L(\mathbb{A})$\\
    \begin{myproof}
        $\mathbb{A}^{c} = (S_A,S_{OA},\cdots , S_A\backslash F_A$
    \end{myproof}

    \nt
    {
        It doesn't work in non deterministic case
        \qs{}
        {Prove the above}
    }
    \nt
    {
        In the nondeterministic case, we first use construction from precious lecture:\\
        $\mathbb{A} \text{ NFA } \leadsto \mathbb{A}^{d} \text{ DFA } L(\mathbb{A}) = L(\mathbb{A}^{d}) \leadsto \mathbb{A}^{dc}$
    }
}

\dfn
{
    Another Construcion
}
{
    $\mathbb{A}^{\cap}\mathbb{B}:\; L(\mathbb{A}^{\cap}\mathbb{B}) = L(\mathbb{A})^{\cap}L(\mathbb{B})=$\\
    $= \{w^{\cap}v\; : \; w \in L(\mathbb{A}), v \in L(\mathbb{B})\}$\\
    Also:\\
    $A^{*}\; : \; L(\mathbb{A^{*}}) = L(\mathbb{A})^{*} =$\\
    $\{ v_0^{\cap}v_1^{\cap}\cdots^{\cap}v_n \; : \; n\in \mathbb{N}\cup \{-1\},\: v_i \in L(\mathbb{A}) \}$
}
