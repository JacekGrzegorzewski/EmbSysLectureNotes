
\chapter{Reminder}

\begin{center}
\begin{tabular}{|c|c|}

\hline
Waves & Particles \\
\hline
Propagation  & trajectory ( position/moment)\\
\hline
Interference & \\
\hline
Diffraction & \\
\hline
Polarization & \\
\hline
\end{tabular}
\end{center}

1900s
$\text{Electromagnetic waves} \leftrightarrow \text{hypothetical particles(???)}$ these hypothetical particles were eventually called photons.
\nt{
Hydrogen atom linear spectrum, photoelectric effect, black body radiation.\\
All important issues in non Newtonian dynamics. WE SHOULD KNOW WHAT ARE THE PROBLEMS WITH THESE QUESTIONS, WHY IS EXPLAINING BLACKBODY RADIATION A PROBLEM?
}

\dfn{Bohr's model of H--atom(1913)}
{
    \begin{itemize}
        \item electron -- follows circular orbits\\
        $\bar{L} = \bar{r}\times \bar{p} \rightarrow L = r m v = n \frac{h}{2\pi}$
            \item $E_n = -\frac{E_0}{n^2}$, $E=13.6\unit{eV}$ \\
                $E_1 = -13.6\unit{eV}\cdots \text{infinitely many excited states}$ \\
                $E_n - E_m = h \frac{c}{\lambda_{hm}}$

    \end{itemize}
}

\nt{1923 -- L. De Broglie\\
Wave -- particle duality of radiation\\
 $\lambda = cT$,  $\mu %% this should be something like a v,but i don't know the letter
= \frac{1}{T}= \frac{c}{\lambda} = pc$, this implies that:\\
$p = \frac{h}{\lambda} $ -- a particle wave duality of matter
}
This matter--wave duality was confirmed by Davisson--Germer experiments
in the latter part of the 1920s.


\section{Quantum mechanics -- 1925}

\begin{itemize}
        \item State of a system(QM) is described by a wave function
        \item EoM --Schrodninger's equation: i\hbar \pdv{\psi}{t} = \hat{H}\psi
        \item Observables -- quantities that can be measured
        
\end{itemize}

\nt{Wave function is usually denoted as: $\psi(x,y,z,t) --$ ????}

\nt{(Classical M.) System \\
\begin{itemize}
        \item The sate of a classical system: $\bar{r}$, $\bar{p}$
        \item Dynamical evolution(Equation of Motion)\\
            $ \pdv{r}{t^2} =   \bar{a}_i = \frac{\bar{F}_i}{m_i}$
\end{itemize}
}

\ex{1D case}
{
    In the 1D case, we can stay that $ \psi(x,t)\mid_{t=t_0}$\\
    In this case $\psi $ means the amplitude of probability.\\
    In that case P[x \in \{x_1,x_2\}] = \int_{x_1}^{x_2} \abs{\psi(x)}^2 dx\\
    $P(x,x+dx) = \rho(x)dx = \abs{\psi(x)}^2dx$
}


