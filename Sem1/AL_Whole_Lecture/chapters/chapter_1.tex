
\chapter{Propositional Logic}
\dfn{Language}
{The language or propositional logic consists of:
        \begin{itemize}
                \item Propositional variables: $p_1,p_2,p_3,\cdots$ where $i \in \{0\}\cup \mathbb{N} but often, these are going to be just lowercase letter$.
                \item Connectives:
                        \begin{itemize}
                                \item $\wedge$ - conjunction
                                \item  $\vee$ - discjnction 
                                \item $\rightarrow$ - implication
                                \item $\leftrightarrow$ - equivalence
                                \item  $\neg$ - negation
                        \end{itemize}
                 \item brackets - (,)
                 \item (sometimes) Two constants : $\mathbb{1},\mathbb{0}$


        \end{itemize}        
}

\dfn{Sentences}
{
 The set of sentences is the smallest set of words(over a desribed alphabet) $\mathcal{S}$ satisfying the following propperties:
        \begin{itemize}
                \item $\forall n \in \mathbb{N} \: p_n \in \mathcal{S}$ 
                \item if $\phi, \psi \in \mathcal{S}$ then $(\phi \wedge \psi)$, $\phi \vee \psi$, $\cdots$,$\neg \phi \in \mathcal{S}$ 
        \end{itemize}


}

\ex{Example of the distinction}{$(p_0\wedge p_1)\rightarrow (p_3 \vee \neg p_2)$ -- is a sentence\\
    $p_0((\neg p_1$ -- is not a sentence}

\nt{Formally, $p_0 \vee p_1$ is not a sentence(not brackets), but we will omit brackets when possible.
Convention: 
\begin{itemize}
        \item $\neg$ is the strongest binding operatore
        \item  $\wedge , \vee $ are next
        \item $\rightarrow, \leftrightarrow $ are the weakest
\end{itemize}
\ex{}{$\neg p \vee q \rightarrow r \vee  \neg p \equiv ((\neg p \vee q) \rightarrow (r \vee \neg p ))$}
}


\dfn{Valuation}
{A valuation is a function $\pi\,:\,\mathbb{N} \longrightarrow \{\mathbb{0},\mathbb{1}\} $ \\
        We can create $\hat{\pi}\,:\,\mathcal{S} \longrightarrow \{\mathbb{0},\mathbb{1}\}$
        \ex{Example valuations}
        {
                \begin{itemize}
                        \item $\hat{\pi}(p_i) = \pi_i$ -- The valuation of a variable is equal to the value of the variable
                        \item $\hat{\pi}(\phi \wedge   \psi) = AND(\hat{\psi},\hat{\phi})$ - The value of the AND function can be read from the table below
                \end{itemize}

                \begin{center}
                \begin{tabular}{|c|c|c|c|c|c|}
                \hline
                        a & b & AND & OR & If & IFF \\
                        0 & 0 & 0 & 0 & 1 & 1 \\
                        0 & 1 & 0 & 1 & 1 & 0 \\
                        1 & 0 & 0 & 1 & 0 & 0 \\
                        1 & 1 & 1 & 1 & 1 & 1 \\
                \hline
                \end{tabular}
                And:
                $NOT(a) = 1-a$

                \end{center}
        }

}
\dfn{Tautology}
{
        A sentence is a tautology $\phi \vDash $ if there exist $\pi$ such that $\tilde{\pi}(\phi) = \mathbb{1}$ 

}
\dfn{Satisfiability}
{A sentence $\phi$ is satisfiable if there exist  $\pi$ suh that  $\tilde{\pi}(\phi)  \mathbb{1}$}
\clm{}{}{$ \vDash iff \neg \phi is not satisfiable$ -- meaning, a sentence is a tautology only if its negation is not satisfiable}
\begin{myproof}

Left right implication:
\begin{enumerate}
        \item Take any valuation $\pi$
        \item by assumption $\tilde{\pi}(\phi) = 1$, which means that  $\neg \phi$ is not satisfiable
\end{enumerate}        
Right left implication:\\
Similarily as before

\end{myproof}
\ex{Tautologies}
{
        \begin{itemize}
                \item $ \vDash \neg (p \vee  q) \leftrightarrow (\neg p \wedge \neg q$
                \item $\neg (p \vee  q) \leftrightarrow (\neg p \wedge \neg q)$
                \item $\neg (p \rightarrow q) \leftrightarrow (\neg p \vee  q$ 
                \item $p \leftrightarrow  q \leftrightarrow ( p \rightarrow q) \wedge (q \rightarrow p)$
                \item $ (p \leftrightarrow q) \leftrightarrow (p \wedge q) \vee ( \neg p \wedge \neg q)$
                \item $p \wedge ( q \vee r) \leftrightarrow (p \wedge q) \vee (p \wedge  r)$
                \item $p \vee  (q \wedge r) \leftrightarrow (p \vee  q) \wedge  ( p \vee  r)$
                
        \end{itemize}
}

\dfn{Conjunctive Normal Form(CNF) and Disjunctive $\cdots $ (DNF)}
{
        A sentence is in CNF if it can be written as a conjunction of disjunctions, formally:
        \begin{equation}
                \phi = \bigwedge_{i = 1}^{n} \bigvee_{j=1}^{k_i} l_{ij} 
        \end{equation}
        where $l_{ij} = p_m \vee l_{ij} = \neg p_m$ for some m.\\
        And equivalently for DNF.
}
\nt
{
        If $ \phi $ is in CNF and $ \phi = (p_0 \vee p_1)\wedge \cdots \wedge  (p_{2n} \vee p_{2n-1})$, then the length of its DNF form will be $\approx 2^{n+1}$ 
}

\dfn{Satisfaction problem -- Boolean satisfying problem}
{
        Given $ \phi$, check if it is satisfiable.
        \nt{If $ \phi $ is in DNF, the problem is trivialy easy, so we assume that $\phi$ is in CNF.\\
        SAT problem is NP complete}
}


