
\chapter{final lecture 2023-06-07}
\section{NFA(DFA) corresponds with Regular Languages(RL)}
\dfn{}
{
    Fix $\Sigma$.  $\omega$--language  $\L \subset \Sigma^{\omega}$ is $\omega$--regular.
    If  $L = \Cup^{n}_i L_i K_i^{\omega}$, where $n \in N$ and  $L_i, K_i$ are regular languages for  $i = 1,2,\cdots ,n$

    Moreover:
    \[
        L_iK_i^{\omega}{\sigma^{\cap}\tau_0^{\cap}\cdots : \tau_1,\cdots \in K_i}
    .\] 
}
\thm{$\omega$--languages accepted by NBA are exactly  $\omega$--regular languages.}
{
    \begin{myproof}
        I: We want to prove that $\omega$--regular language L is accepted by some NBA  $\mathbb{A}$.\\
            Task: Find $\mathbb{A}$ s.t. $L_\omega(\mathbb{A}) = L$\\
            \begin{enumerate}
                \item Union is easy\\
                \item Consider $L_iK_i^{\omega}$ \\
                    We know that there are $\mathbb{B}, \mathbb{C}$ NFA s.t. $L(\mathbb{B}) = L_i, L(\mathbb{C})=K_i$
            \end{enumerate}
       II: Assume $\mathbb{A}$ is NBA, $\mathbb{A} = (S,S_0,\Sigma,\delta,F)$ \\
       $L_\omega(\mathbb{A}) = \bigcup_{s\in S_0}\bigcup_{f\in F} L((S,\{s\},\Sigma,\delta,\{f\}))L((S,\{f\},\Sigma,\delta,\{f\}))^{\omega}$
    \end{myproof}
}

\section{Connection between logic and automata}
More specifically, between LTL logic and NBA.
\begin{myproof}
    Let: $ \phi \in \mathcal{L}_{LTL}(\mathcal{P})$, wlog $\mathcal{P}$ is finite, $\mathcal{P} = \text{var}(\phi)$ = a set of prepositional variables which appear in $\phi $.\\
    We want to find(describe):
    \begin{equation}
        \mod(\phi ) = \{ \pi \in P(\mathcal{P})^{\omega} \;:\; \pi,0 \vDash \phi\}
    \end{equation}
    Now, set $\Sigma = P(\mathcal{P})$ (a finite) alphabet, and now  $\mod{\phi } \subset \Sigma^{\omega}$
\end{myproof}
\thm{}
{
    Let $\phi \in \mathcal{L}_{LTL}(\mathcal{P})$, $\mathcal{P} = var(\phi )$. Set $\Sigma = P(\mathcal{P})$\\
    Then there exists $\mathbb{A}_\phi$ NBA s.t.\\
     \[
         L_\omega(\mathbb{A}_\phi) = \mod{\phi}
    .\] 
}
\qs{}
{How to construct a "small" $\mathbb{A}_\phi$}

\ex{$\phi = \Diamond p \wedge \Diamond q$}
{
    First, $\Sigma = P(\{p,q\}) = \{\emptyset,\{p\},\{q\},\{p,q\}\}$\\
    \begin{equation}
        \begin{aligned}
            $\mod{\phi } &= \{ \pi \in \Sigma^{\omega}\;:\; \pi, 0 \vDash \Diamond p \wedge \Diamond q \}&\mbbox{}\\[1.25ex]$ 
            \mod{\phi }&= \{\pi\in\Sigma^{\omega} : (\exists k,l )(\pi(k)\ni p \wedge \pi(l) \ni q)\}&\mbox{}\\[1.25ex]
            
            
        \end{aligned}
    \end{equation}
}

%Promela and spin, a programming language for LTL logic
