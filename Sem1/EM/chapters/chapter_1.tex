
\chapter{Interfaces 2023-05-09}


\section{CAN}
\begin{itemize}
        \item Differential wiring 
        \item Not very fast
        \item Very strong CRC
        \item Acknowledge after every received frame
        \item 11 bit addressing and arbitration, sometimes 26. Lower ID's get priority when they appear simultaneously on the line.
        \item  Acceptance Filters for sorting which messages we want to receive. Can be configured either for single addresses, or for ranges.
        \item THIS FILTERING MEANS WE CAN SOMETIMES SEND NO DATA IN A FRAME.
        \item Very reliable
        \item Bit-stuffing every 5 consecutive zeros. Used for synchronization 
        
\end{itemize}

\subsection{Network topology}
Usually microcontroller needs to have a CAN controller, it's very difficult to write one by yourself.
This by itself is not enough, one also needs a transceiver. This transceiver sends the messages to the
CAN BUS, which is usually a differential line.
Each can controller has a set of mailboxes. 

\nt
{
    FIFO vs MAILBOX\\
    MAILBOX has priorities assigned to messages, FIFO does not.
    MAILBOX then is not a pipe. Priorities are assigned according to the address of a message.
}

\clearpage
\section{USB}
\begin{itemize}
        \item A serial, 2--wire, differential interface.
        \item supposedly universal
        \item Works up to 5 meters
        \item Single Master
        \item Asynchronous transfer controlled by the master
        \item Half duplex, point to point transmission
        \item Variable, but mostly fast speeds. For most microcontrollers only FS mode ($\frac{12Mb}{s}$) is supported. Nowadays sometimes also HS mode($\frac{480Mb}{s}$)
\end{itemize}

\subsection{Topology}
\begin{itemize}
        \item "Tiered star" topology, which in human terms means a tree.
        \item 2 types of peripherals, Hubs and Fnctions. Up to 127 of those
        \item  Master / Slave
        \item Variable transmission speeds
        \item Bus is self powered. Provides 4.35 V at either 100mA or 500mA. It's very noisy.

\end{itemize}

\subsection{Protocol}
\begin{itemize}
        \item Host based token polling
        \item Very robust. Handshakes, low physical error rate, optional CRC.
        \item Bandwidth and latency prenegotiated. THIS MEANS WE CAN SEND TIME SENSITIVE DATA
        \item Unique address is assigned after connecting to the network. Addresses are not inherent to devices.
\end{itemize}
\subsection{Transfer}
\begin{itemize}
        \item Several endpoints
        \item Several types of transfers
        \item Transfer is scheduled by the host
        \item Isochronous transfer, which means each frame is sent at specific time instances.
        
\end{itemize}

Device detection works as follows. The host pulls the line down, and a new device pulls it up after its connected to the network. Once this is detected its
assigned an address by the host. The host also configures the device.
\nt
{
    USB has a lot of connectors to choose from. The cable is somewhat standard however. There are 2 power conductors, and the twisted, unshielded differential pair.
    The shielding is done over all 4 cables together with a foil and braiding.\\
    In newer USB 3.0 cables, the matter is more complicated. The connectors however are symmetrical.
}

\subsection{practice}
FT232R USB URAT IC -- a single chip transceiver.
\nt
{
    It's better to use a single than split power supply. This is because if its split, we can potentially connect an unpowered device to the network. This can 
    result in improper configuration, which is difficult to debug. 
}
OTG USB -- On The Go USB. This is a feature which allows to dynamically change from a host to a slave and vice versa.


\section{Ethernet}
Ethernet is a protocol for connecting computers in a local area network. Invented in California in Xerox.
