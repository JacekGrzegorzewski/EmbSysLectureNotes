\chapter{Power consumption???? 2023-05-23}

\section{Switching power consumption}

\nt
{
    $P = V^{2}\times f \times C$
}
Because the power is dependant on the square of the voltage, we would like to lower it, but this make our system more susceptible to noise.

\section
{
    Sources of power consumption
}
\subsection{Static}

\subsection{Dynamic}


\chapter{Power reduction of MCU}
If we want to use an MCU in low power aplications, it must(or would like to have) the following:

\begin{itemize}
        \item Sleep modes
        \item Automatic wake-up on time intervals -- usually done with watchdog, a low frequency independent clock, but with low accuracy.
        \item Power domains
        \item RAM retention during standby
        \item Power monitoring
\end{itemize}

\section{Sleep modes}
Main sleep modes:
\begin{itemize}
        \item Active -- Everything's running
        \item Idle -- CPU Clock off -- nanoseconds to switch to active
        \item Standby -- CPU Clock OFF, some peripherals on, but RAM is retained -- microseconds to go back to active
        \item Backup -- RAM is not kept, so we need to backup the data before going into sleep. Since RAM is not kept,
            it is not powered, so there are no static losses. This is the most efficient mode, but we need to restart
            the processors when waking up, so it's slow. -- hundreds of microseconds to go back to active
\end{itemize}
\section{Power domains}
\section{Power monitoring}
Necessary to ensure that we don't drop below a certain voltage threshold, below which CMOS logic doesn't work properly. 
\nt{
VERY IMPORTANT IN BATTERY POWERED SYSTEMS!!!

}
\section{Power reduction of supporting electronics}
\begin{itemize}
        \item Maximize impedance in current paths - increases susceptibility to nosie
        \item Minimize impedance in high--speed switching paths 
        \item Minimize leakage currents
        \item Minimize operating duty cycles
        
\end{itemize}



\chapter{Pipelining}
There are 3(5) stages of running an instruction on a CPU:
\begin{itemize}
        \item Fetch -- get instruction
        \item Decode -- decode and get data
        \item Execute -- execute instruction
        \item sometimes Memory access
        \item sometimes Write Back
\end{itemize}

The pipeline's task is to keep all of these stages working all the time.



