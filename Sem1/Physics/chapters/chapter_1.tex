
\chapter{Reminder}

\begin{center}
\begin{tabular}{|c|c|}

\hline
Waves & Particles \\
\hline
Propagation  & trajectory ( position/moment)\\
\hline
Interference & \\
\hline
Diffraction & \\
\hline
Polarization & \\
\hline
\end{tabular}
\end{center}

1900s
$\text{Electromagnetic waves} \leftrightarrow \text{hypothetical particles(???)}$ these hypothetical particles were eventually called photons.
\nt{
Hydrogen atom linear spectrum, photoelectric effect, black body radiation.\\
All important issues in non Newtonian dynamics. WE SHOULD KNOW WHAT ARE THE PROBLEMS WITH THESE QUESTIONS, WHY IS EXPLAINING BLACKBODY RADIATION A PROBLEM?
}

\dfn{Bohr's model of H--atom(1913)}
{
    \begin{itemize}
        \item electron -- follows circular orbits\\
        $\bar{L} = \bar{r}\times \bar{p} \rightarrow L = r m v = n \frac{h}{2\pi}$
            \item $E_n = -\frac{E_0}{n^2}$, $E=13.6eV$ \\
                $E_1 = -13.6 eV\cdots \text{infinitely many excited states}$ \\
                $E_n - E_m = h \frac{c}{\lambda_{nm}}$

    \end{itemize}
}

\nt{1923 -- L. De Broglie\\
Wave -- particle duality of radiation\\
 $\lambda = cT$,  $\mu %% this should be something like a v,but i don't know the letter
= \frac{1}{T}= \frac{c}{\lambda} = pc$, this implies that:\\
$p = \frac{h}{\lambda} $ -- a particle wave duality of matter
}
This matter--wave duality was confirmed by Davisson--Germer experiments
in the latter part of the 1920s.


\section{Quantum mechanics -- 1925}

\begin{itemize}
        \item State of a system(QM) is described by a wave function
        \item EoM --Schrodninger's equation: $i\hbar \pdv{\psi}{t} = \hat{H}\psi$
        \item Observables -- quantities that can be measured
\end{itemize}

\nt{Wave function is usually denoted as: $\psi(x,y,z,t) --$ ????}

\nt{(Classical M.) System \\
\begin{itemize}
        \item The sate of a classical system: $\bar{r}$, $\bar{p}$
        \item Dynamical evolution(Equation of Motion)\\
            $ \pdv{r}{t^2} =   \bar{a}_i = \frac{\bar{F}_i}{m_i}$
\end{itemize}
}

\ex{1D case}
{
    In the 1D case, we can stay that $ \psi(x,t)\mid_{t=t_0}$\\
    In this case $\psi $ means the amplitude of probability.\\
    In that case $P[x \in \{x_1,x_2\}] = \int_{x_1}^{x_2} \abs{\psi(x)}^2 dx$\\
    $P(x,x+dx) = \rho(x)dx = \abs{\psi(x)}^2dx$
}

\qs{Homework}
{
    Read Ch.  p. 160\\
    "Concepts in MP" A. B.
}

\chapter{Quantum Mechanics 2023-05-09}

\dfn{Classical mechanics}
{
    Concepts in classical mechanics
    \begin{itemize}
            \item A state of a system: $\bar{r}$, $\bar{p}$ 
            \item Evoltion(EoM): $m \odv{r}{t}{t} = \bar{F}$, $\bar{r}(t_0), \bar{p}(t_0)$
            \item The above describes the state as $\bar{r}(t)$ and $\bar{p}(t)$, along with: $E_k=\frac{mv^{2}}{2} = \frac{p^{2}}{2m}$, $\bar{L}=\bar{r}\times \bar{p}$, $U(\bar{r})$
            
    \end{itemize}
}
\dfn{Quantum mechanics}
{   Quantum mechanics were formulated around 1925 / 26 by, among others, E. Schrodninger, N. Bohr, M .Born, P.A.M Dirac.
    As oposed to classical mechanics, QM often goes against our intuitions, which makes it somewhat difficult to understand.
    Since its such a deep and complex topics, we're not going to develop a background for it all. Instead, we're going to
    focus on one specific concept in that area, namely a Quantum Harmonic Oscillator.
}

\dfn{Wave function}
{
    One of the most important objects in QM is called the wave function. It is difficult to interpret it physically, but it describes the probability density of the chance to find a particle in a given position.
    \begin{equation}
        \psi (\bar{r},t)\mid_{t=t_0}:\; P(\Delta V) = \int_{\Delta V} \norm{\psi}^{2}dv
    \end{equation}
    Where this function is considered to be normalized, meaning that its integral over the whole space is equal to 1.

}

\dfn{Evolution("EoM")}
{
    The evolution of a quantum system is described by the \textit{Schrodninger's equation}:
    \begin{equation}
        \label{schrod}
        i\hbar \pdv{\psi }{t} = -\frac{\hbar^{2}}{2m}( \pdv{}{x,x} + \pdv{}{y,y} +\pdv{}{z,z})\psi  + U(\bar{r})\psi 
    \end{equation}
    The solution of which is a wave function
}
\clearpage
\section{Postulates of QM}
QM postulates the following statements:
\begin{enumerate}
    \item The state of the system is described by a wave function $\psi $
    \item The evolution of the system follows equation \ref{schrod}
    \item For every observable quantity A, there is a corresponding operator $\hat{A}$:  $A(\bar{r},\bar{p}) \longrightarrow \hat{A}(\hat{\bar{r}},\hat{\bar{p}})$
\end{enumerate}
\dfn{Operator}
{
    An operator $\hat{A}$ is a mapping from the space of wave functions into some other function space. As an example:
    \[
        \hat{p_x} = -i\hbar\pdv{}{x}
    .\]         
    More formally, it is a mapping from the Hilbert space to the Hilbert space.
}

\qs{}
{
    Find the kinetice energy operator in QM
}
\sol $\hat{T} = \frac{\hat{p}^{2}}{2m} = \frac{1}{2m}(\hat{p}_x^{2} + \hat{p}_y^{2} + \hat{p}_z^{2}) = \frac{ \hbar}{2m}( \pdv{}{x,x} + \pdv{}{y,y} + \pdv{}{z,z}) $ 

\nt
{
    Using operators we can rewrite \ref{schrod} as:
    \[
    i\hbar \pdv{}{t}\psi = \hat{H}\psi = (\hat{T}+\hat{U})\psi 
    .\] 
}

\dfn{Commutators}
{
    A commutator is the operator: 
}

\clm{}{}{QM is an algebra of non--commuting variables}


