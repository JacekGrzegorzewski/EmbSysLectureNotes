
\chapter{Input -- State Linearization of SISO systems 2023-05-11}

\dfn{A single input nonlinear system}
{
    Let $f, g$ be smooth vector fields on $\mathbb{R}^{n}$, then:
     \begin{equation}
        \Sigma:\; \dot{x} = f(x) + g(x)u
    \end{equation}
    is said to be input--output linearisable if there exists a region $\Omega \subset \mathbb{R}^{n}$,
    a differomorphism $T\,:\,\Omega \longrightarrow \mathbb{R}^{n} $ and a nonlinear feedback
    control law:
    \begin{equation}
        u(x) = \alpha(x) + \beta(x)v
    \end{equation}
    such that the new state variables $z = T(x)$  and the new input $v$ satisfy
    a linear time invariant relation:
     \begin{equation}
        \dot{z} = Az + bv
    \end{equation}
    where:
    \[
    A = \begin{bmatrix}
        \mathbb{0} & I_{n-1}  \\
        0 & \mathbb{0}^{T}
    \end{bmatrix},\; b = \begin{bmatrix}
        \mathbb{0}  \\
        1
    \end{bmatrix}
    .\] 

    \nt
    {
        We call:
        \begin{itemize}
                \item z -- a linearising state
                \item (A,b) -- a linear companion form
                
        \end{itemize}
    }

}

\thm{}
{
    The nonlinear system:
    \begin{equation}
        \dot{x} = f(x) + g(x)u
    \end{equation}
    with f(x),g(x) being smooth vector fields is input--state linearisable iff there exists a region $\Omega$ such that the following
    conditions hold:
    \begin{itemize}
        \item The vector fields $\{g, ad_fg,\cdots ,ad^{n-1}_fg\}$ are linearly independent
        \item The set $\{g, ad_fg,\cdots ,ad^{n-2}_fg\}$ is inductive in $\Omega$
    \end{itemize}
    \ex{}
    {
        Let:
        \begin{equation}
            \dot{x} = Ax+bu, \; b\in \mathbb{R}^{n\times1}
        \end{equation}
        Then:
        \begin{equation}
            \begin{cases}
                f(x) = Ax\\
                g(x) = b
            \end{cases}
        \end{equation}
        We calculate the Lie brackets:
        \begin{equation}
            \begin{aligned}
                ad_fg&=[f,g]&\mbox{}\\[1.25ex]
                [f,g]&= \pdv{g}{x}f - \pdv{d}{x}g&\mbox{}\\[1.25ex]
                [f,g]&= \pdv{}{x}b(Ax) - Ab&\mbox{}\\[1.25ex]
                [f,g]&=-Ab&\mbox{}\\[1.25ex]
            \end{aligned}
            \begin{aligned}
                ad_f^{2}g&=[f,[f,g]]&\mbox{}\\[1.25ex]
                [f,[f,g]]&=[Ax. -Ab]&\mbox{}\\[1.25ex]
                [f,[f,g]]&= \pdv{}{Ab}b(Ax) - \pdv{}{x}(Ax)(-Ab)&\mbox{}\\[1.25ex]
                [f,[f,g]]&=A^{2}b&\mbox{}\\[1.25ex]
            \end{aligned}

        \end{equation}
        By continuing we find the following pattern:
        \begin{equation}
            ad_f^{n}g=(-A)^{n}b
        \end{equation}
        so the condition:
        \[
            [g,ad_fg,\cdots ,ad_f^{n-1}g] = [b, -Ab, \cdots \pm A^{n-1}b]
        .\] 
        So in this case it is equal to controllability of a linear system.
        Intuitively then, these conditions generalize the concept of controllability to non linear systems.
    }

}






\begin{myproof}
    Proof of the previous theorem, beginning with the necessity condition.
    To prove necessity we will assume that a system in input-output linearisable, and try to conclude the conditions stated in the theorem.\\
    There exists $u(x) = \alpha(x)+\beta(x)v$ and there exists  $T\: : \: z = T(x)$ such that:
    \[
    \begin{bmatrix}
        \dot{z}_1 \\
        \vdots\\
        \dot{z}_n
    \end{bmatrix} = \begin{bmatrix}
    \mathbb{0} & I_{n-1} \\
    0 & \mathbb{0}^{T}
    \end{bmatrix}\begin{bmatrix}
        z_1  \\
        \vdots\\
        z_n
    \end{bmatrix}
    + \begin{bmatrix}
        \mathbb{0}  \\
        1
    \end{bmatrix} v
    .\] 
    Then we can write:
    \[
    \begin{cases}
        z_1 = T_1(x)\\
        z_2 = T_2(x)\\
        \vdots \\
        z_n = T_n(x)
    \end{cases}
    .\] 
    From which we get:
    \[
        \begin{cases}
        \dot{z}_1 = \odv{}{t}T_1(x) = \pdv{}{x}T_1\dot{x} = \pdv{}{x}T_1(f(x)+g(x)u) = z_2=T_2(x)  \\
        \dot{z}_2 = \odv{}{t}T_2(x) = \pdv{}{x}T_2\dot{x} = \pdv{}{x}T_2(f(x)+g(x)u) = z_3=T_3(x) \\
        \vdots \\
        \dot{z}_{n-1} = \odv{}{t}T_{n-1}(x) = \pdv{}{x}T_{n-1}\dot{x} = \pdv{}{x}T_{n-1}(f(x)+g(x)u) = z_{n}=T_{n}(x) \\
        \dot{z}_n = v
        
            
        \end{cases}
    .\] 
\end{myproof}


\section{Reminder 2023-05-29}
Given:
\[
    \begin{cases}
        \dot{x} = f(x) + g(x)u\\
        y = h(x)
    \end{cases}
.\] 
If $L_gL_f^{r-1}h(x) \neq 0$ then $u = \frac{1}{L_gL_f^{r-1}h(x)}(-L_f^{r}h(x) + v)$ which yields $y^{(r) = v}$.

\dfn{Relative degree}
{
    The SISO system is sad to have the relative degree r in a region of $\Omega$ iff  $\forall x \in \Omega L_gL_f^{i}h(x) = 0\; \forall i 0 \le i \le r-1,\; L_gL_f^{r-1}h(x) \neq 0  $ 

}
\dfn{Normal form}
{
    The normal form of the system can be written as:
    \begin{equation}
        \begin{aligned}
            \begin{bmatrix}
                \dot{\xi_1}   \\
                \vdots\\
                \dot{\xi}_{r-1}\\
                \dot{\xi}_r
            \end{bmatrix} &= \begin{bmatrix}
                \xi_2 & \cdots  \\
                \vdots & \\
                \xi_r \\
                a(\xi,\eta) - b(\xi,\eta)u
            \end{bmatrix} \\
            \dot{\eta} &= \omega(\xi,\eta)\\
            y &= \xi_1
        \end{aligned}
    \end{equation}
    where $\xi = [h_1, L_f h,\cdots, L_f^{r-1}]$, and $\eta = [\eta_1,\cdots ,\eta_{r-1}]$
}
\nt{MISSING DEFINITONS}

With this transformation the nonlinear system is transformed to the normal form with:
\begin{equation}
    \begin{cases}
a(\xi,\eta) = L_f^{r}h(x) = L_f^{r}h(\phi^{-1}(\xi,\eta))\\
b(\xi,\eta) = L_gL_f^{r-1} h(x) = L_gL_f^{r-1}h(\phi^{-1}(\xi,\eta))
    \end{cases}
\end{equation}


\mlenma{}
{
Since $L_g\xi_i = 0, i = 1,2,\cdots ,r-1$ we can conclude that the scalar functions $\eta_j,\; j=1,2,\cdots ,n-r$ need to be chosen such that:
\begin{equation}
    L_g\eta(x) = 0 \text{ and rank}[[ \pdv{\xi_1}{x}]^{T},\cdots, [ \pdv{\xi_r}{x}]^{T}[ \pdv{\eta_1}{x}]^{T}, \cdots [ \pdv{\eta_{n-r}}{x}]^{T}] = n \forall x \in \Omega 
\end{equation}

}

\ex{}
{
    Let:
    \begin{equation}
        \begin{aligned}
        \begin{bmatrix}
            \dot{x}_1  \\
            \dot{x}_2 \\
            \dot{x}_3
            \end{bmatrix} &= \begin{bmatrix}
            -x_1 \\
            x_1x_2\\
            x_2
        \end{bmatrix}
        + \begin{bmatrix}
            e^{x_2}\\
            1\\
            0
        \end{bmatrix} u\\
        y &= x_3\\
    \end{aligned}
    \end{equation}
    The determination of the relative degree:
    \begin{equation}
        \begin{aligned}
            \dot{y} &= \dot{x}_3 &= x_2\\
            \ddot{y} &= \dot{x}_2 &= x_1x_2+u
        \end{aligned}
    \end{equation}
Thus the system has relative degree 2.
\begin{equation}
    \begin{aligned}
        \xi_1 &= y &= x_3\\
        \xi_2 &= L_f h(x) &= \pdv{h}{x}f\\
        \xi_2 &= \pdv{x_3}{x}f
    \end{aligned}
\end{equation}
}
We have $z = \Phi(x) = \begin{bmatrix}
    \xi_1 & \xi_2 & \eta 
\end{bmatrix}^{T}$
then:
\begin{equation}
    \pdv{z}{x} = \begin{bmatrix}
        0 & 0 & 1 \\
        0 & 1 & 0 \\
        1 &-e^{x_2} &0
    \end{bmatrix}
\end{equation}
with its rank being 3 for any x, since it's not rank deficient , we can find an inverse transformation:
\begin{equation}
    \begin{cases}
        x_1 = -1 +\eta + e^{\xi_2}\\
        x_2=\xi_2\
        x_3=\xi_1
    \end{cases}
\end{equation}
With this set of coordinates(state transformation) the syste dynamics is put into a normal form:
\begin{equation}
    \begin{cases}
        \dot{\xi}_1 - \xi_2\\
        \dot{\xi_2} = (-1+\eta+e^{\xi_2})\xi_2+u\\
        \dot{\eta} = (1-\eta + e^{\xi_2)(1-\xi_2e^{\xi_2})\\
            y = \xi_1
    \end{cases}
\end{equation}
Let $u = -(-1+\eta+e^{\xi_2})\xi_2 + v$ which gives us:
\begin{equation}
    \begin{cases}
        \dot{\xi}_1 = \xi_2\\
        \dot{\xi}_2 = v\\
        \dot{\eta} = (1-\eta + e^{\xi_2)(1-\xi_2e^{\xi_2})\\
            y = \xi_1
    \end{cases}
\end{equation}

\dfn{The zero--dynamics}
{
    The zero--dynamics of the non--linear system:
    \[
    \begin{cases}
        \dot{x}=f(x)+g(x)u\\
        y = h(x)
    \end{cases}
    .\] 
    is the dynamics of the system subject to the constraint that the output is identically zero.\\
    $y = 0 \rightarrow \dot{y},\ddot{y},\cdots =0$\\
    $x_0 \in M^{*} = \{ x \mid h(x) = L_fh(x) = \cdots =L_f^{r-1}h(x) = 0 \}$\\
    $y^{(r)}(t) = 0 \rightarrow u^{*} = \frac{L_f^{r}h(x)}{L_gL_f^{r-1}h(x)}$ 
    When we express the system in the normal form then:
    \begin{equation}
        \begin{cases}
            \dot{\xi} = 0\\
            \dot{\eta} = w(0,\eta),\; u^{*} = -\frac{a(0,\eta(t)}{b(0,\eta(t)}
        \end{cases}
    \end{equation}
}

\ex{Continued}
{
    $\dot{\eta} = (1-\eta - e^{\xi_2})(1+\xi_2e^{\xi_2})$ \\
    Its zero dynamics are defned by letting $\xi_1=0, \xi_2=0$.\\
    $\dot{\eta}= -\eta\;\; (\eta(t) = \e^{-t}\eta(0)$
}
\dfn{}
{
    The nonlinear system $\dot{x} = f(x)+b(x)u,\; y = h(x)$ is said to be asymptotically minimum phase if  its zero dynamics is
    asymptotically stable
}

\thm{Stabilization and tracking}
{
    Assume that the system:
    \[
    \dot{x} = f(x) + g(x)u\\
    y = h(x)
    .\] 
has relative degree r and its zero dynamics is asymptotically stable. Let $\mu(s) = s^{r + \alpha_{r-1}s^{r-1}+\cdots +\alpha_0}}$ be a hurwitz polynomial. Then thestate feedback law:\\
$
u(x) = \frac{1}{L_gL_F^{r-1}h(x)}[-L_f^{r}h(x) - \alpha_{r-1}L_f^{r-1}h(x)\cdots -\alpha_0h(x)]
$
Leads to aa locally asymptotically stable closed loop system
