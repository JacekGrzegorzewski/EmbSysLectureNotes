
\chapter{Feedback Linearization}
\section{Intuitive concepts}
This chapter focuses on the consequences of applying Lyapunov's first method to the feedback loop.
Linear control is very well developed field, so moving from nonlinear to linear world is desirable.
The drawback however is that such a control scheme is local, meaning that if we deviate far enough
from an equilibrium point everything breaks down. A solution to this problem, is the use of 
auxiliary input in the feedback loop.


\ex{Flow from a water tank}
{
    \nt{DESCRIPTION OF TO BE INCLUDED PICTURE\\
        We have a water tank, with flow u(t) into the tank,
        and the area of the water line A(h) marked. Additionally,
        distance h from the water level to the output faucet is marked as h,
        with the area of the output faucet being a constant a}
        The equation ascribing the flow of water through the facet is the following:
        \begin{equation}
            \label{FL:ex}
            \frac{d}{dt}[\int_0^h A(\bar{h})d\bar{h}] = u(t) - a\sqrt{2gh}
        \end{equation}
        Since $ \frac{d}{dt}[\int_0^h A(\bar{h})d\bar{h}]$ is equal to $A(h)\dot{h}$,
        we have:
        \begin{equation}
            \label{FL:ex2}
            A(h)\dot{h} = u(t) - a\sqrt{2gh}
        \end{equation}

        Let us set $u(t) = a\sqrt(2gh)+A(h)v$, substituting into  \ref{FL:ex2} we obtain:
        \begin{equation}
            \label{FL:ex_3}
            A(h)\dot{h} = a\sqrt{2gh} + A(h)v - a\sqrt{2gh}
        \end{equation}
        After subtracting like terms we obtain $\dot{h} = v$(A linear equation!!!).
        Based on this we can create a control law:
        \begin{equation}
            \label{FL:excl}
            v = \alpha(h_{ref}-h), \; \alpha > 0
        \end{equation}
        Ergo:\\
        $\dot{h}+\alpha h = \alpha h_{ref}$\\
        An easily solvable linear equation.
        This procedure is called perfect linearization
}

\dfn{Companion form of a system}
{
    A system is said to be in a companion form, if it can be described as follows:
    \begin{equation}
        \label{FL:dfn_1}
        \Sigma : \begin{bmatrix}
             \dot{x_1}  \\
             \dot{x_2}\\
             \vdots\\
             \dot{x_{n_-1}}\\
             \dot{x_{n}}
        \end{bmatrix}
        = \begin{bmatrix}
            x_2  \\
            x_3  \\
             \vdots\\
            x_n  \\
            f(x)+b(x)u  
        \end{bmatrix}
    \end{equation}

    \begin{equation}
        \label{FL:dfn_2}
        C :  u = \frac{1}{b(x)}(v-f(x))
    \end{equation}

    \begin{equation}
        \label{FL:dfn_3}
        Cv :  v = -k_0x-\cdots -k_{n-1}x^{n-1}
    \end{equation}

    Ultimately we can write:
    \begin{equation}
        \label{FL:dfn_4}
        \Sigma +C +C_v : \begin{bmatrix}
            \dot{x_1}  \\
            \dot{x_2}  \\
            \vdots \\
            \dot{x_n}  \\
        \end{bmatrix}
        = \begin{bmatrix}
            0 & 1& 0& \cdots &0  \\
            \vdots &  & \ddots&  &\vdots  \\
            0 &  & \ddots& 0 & 1  \\
            -k_0 &  & \cdots &  & -k_{n-1}  \\
        \end{bmatrix}
        = \begin{bmatrix}
            x_1  \\
            x_2  \\
            \vdots \\
            x_n  \\
        \end{bmatrix}
    \end{equation}
    HMS control law guarantees that
        $ \begin{bmatrix}
            x_1  \\
            x_2  \\
            \vdots \\
            x_n  \\
        \end{bmatrix} \rightarrow 0$
        \\ Suppose that we want that $x \rightarrow x_{ref}$ where $x_r$ is a differentiable time dependant function.
        In such a case, let: $e(t) = x(t) - x_r(t)$\\
        $v = x_r^{(n)} - k_{n-1}e^{(n-1)}-\cdots -k_{0}e = 0$\\
        $\Sigma+ C +V : e^{(n)}+k_{n-1}e^{(n-1)} + \cdots + k_0e=0$
}

\section{Input-State Linearization}
From the equations:
\begin{equation}
    \begin{cases}
          \dot{x} = f(x,u)\\
         z = w(x)
    \end{cases} \rightarrow 
    \begin{cases}
        \dot{z}=\bar{f}(z,u)\\
        u = g(z,v)
    \end{cases} \rightarrow  \dot{z} = Az +bv
\end{equation}
Where we assume that $w(x)$ is an invertible function. We can then invert the process, and find z as a function of v.

\ex{}
{
    \begin{equation}
        \label{ISL:ex_1}
        \begin{bmatrix}
            \dot{x}_1  \\
            \dot{x}_2
        \end{bmatrix}
        = \begin{bmatrix}
            -2x_1+ax_2+\sin(x_1)  \\
            -x_2\cos(x_1)+u\cos(2x_1)  
        \end{bmatrix},\; x_e=\begin{bmatrix}
            0\\0
        \end{bmatrix}
        , u_e=0
    \end{equation}
    From this we obtain:
    \begin{equation}
        \label{ISL:ex_2}
       \begin{cases}
           z_1=x_1\\
           z_2=ax_2+\sin {x_1}
       \end{cases}
       \rightarrow 
       \begin{cases}
           x_1=z_1\\
           x_2=\frac{z_2-\sin{z_1}}{a}
       \end{cases}
    \end{equation}
    And:
    \begin{equation}
        \label{ISL:ex_3}
        \begin{cases}
            \dot{z}_1 = -2z_1+z_2\\
            \dot{z}_2 = (\sin{z_1}-2z_1)\cos{z_1}+ua\cos{2z_1}
        \end{cases}
    \end{equation}
    \nt{We obtain the above by noticing that:
        \begin{equation}
            \begin{bmatrix}
                \dot{x}_1 \\ \dot{x}_2
            \end{bmatrix}
            =
            \begin{bmatrix}
                -2z_1 +z_2 \\
                \frac{\sin{z_1}\cos{z_1}-z_2\cos{z_1}}{a} +u\cos{2z_1}
            \end{bmatrix}
        \end{equation}
    }

    This ultimately results in:
    \begin{equation}
        C : u = \frac{1}{acos(2z_1}[]-(\sin{z_1}-2z_2)\cos{z_1}+v]
    \end{equation}
   This control law linearises the system, giving us:
   \begin{equation}
       \Sigma +C : \begin{bmatrix}
           \dot{z}_1 \\\dot{z}_2
       \end{bmatrix}
       = \begin{bmatrix}
           -2 & 1 \\
           0 & 0
       \end{bmatrix}
       \begin{bmatrix}
           z_1 \\ z_2
       \end{bmatrix}
       + \begin{bmatrix}
           0 \\ 1
       \end{bmatrix}v
   \end{equation}
   Lets make: $v = \begin{bmatrix}
       -k_1 & -k_2 
   \end{bmatrix}
   \begin{bmatrix}
       z_1 \\ z_2
   \end{bmatrix}
   $
   Then we have:
   \begin{equation}
       \begin{bmatrix}
           \dot{z}_1 \\\dot{z}_2
       \end{bmatrix}
       = \begin{bmatrix}
           -2 & 1 \\
           -k_1& -k_2
       \end{bmatrix}
       \begin{bmatrix}
           z_1 \\ z_2
       \end{bmatrix}
   \end{equation}
    If we set $k_1=0, k_2=2$, the system will be asymptotically stable.

    The final control law:
    \begin{equation}
        \begin{cases}
            u(z) = \frac{1}{a\cos{2z_1}}[-(\sin{z_1}-2z_1)\cos{z_1}-2z_2]\\
            u(x) = \frac{1}{a\cos{2x_1}}[-(\sin{z_1}-2z_1)\cos{z_1}-2z_2]


        \end{cases}
        
    \end{equation}

}

